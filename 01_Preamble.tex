\maketitle
\makedeclaration


% \begin{thought}
% \begin{flushleft}

% \textit{The timeless task of architecture is to create embodied and lived existential metaphors that concretise and structure our being in the world. Architecture reflects, materialises and eternalises ideas and images of ideal life. Buildings and towns enable us to structure, understand and remember the shapeless flow of reality and, ultimately, to recognise and remember who we are. Architecture enables us to perceive and understand the dialectics of permanence and change, to settle ourselves in the world, and to place ourselves in the continuum of culture and time.}\newline
% Juhani Pallasmaa 
% \end{flushleft}
% \end{thought}



\begin{abstract} % 300 word limit
How should one go about designing for interactions with large programmable
electronic displays? Part of the challenge is that there are currently only a
handful of large interactive surfaces in existence, and so there is much to learn
from each attempt to deploy interactive systems. Hence, the work outlined in
this chapter contributes to Human–Computer Interaction (HCI) research as well
as architectural research by juxtaposing existing interaction frameworks. These
frameworks are concerned with the awareness of spaces mediated through information
and communications technology (ICT), participants and their actions
within these spaces as well as the physical properties of these spaces, which frame
these interactions, and are surrounded by the physical built environment. We introduce
the notion of media architectural interfaces (MAI), which is then supported
through the description of the design, deployment and evaluation of two design
studies, namely VEIV London and SCSD Sao Paulo. Finally, we discuss the multilayered
interaction frameworks with regard to the conducted design studies and
summarise the relevant communalities of these design studies in a taxonomy. The
aim of this categorisation is to provide design implications for future MAI projects.
Ultimately, this may support the design and development of novel and sustainable
interactive systems in the domains of urban screens, media facades and
media architecture.
\end{abstract}



% \begin{preface}
% This research commenced during my attendance in the MSc Adaptive Architecture and Computation programme at UCL The Bartlett School of Architecture in 2010/11. As part of the course module ‘'Cities as Interface" run by Ava Fatah gen. Schieck I came up with the physical ‘I like’ button. This novel tangible interface, that until then existed only in digital space, allowed people for the first time to comment on experiences in situ and share them with their remote friends on digital social networks in real-time \cite{Behrens2011} (Behrens, 2011). Back then I was trying to explore this innovative research not only from an academic perspective but also aimed to investigate how we could turn this research and its technology into a business. In this respect I was taking part in the UCL Advance ''StartUp Summer", in collaboration with YouGov. In a team we developed a business plan for a venture that would implement feedback sensors in businesses to improve the business-consumer relationship. The result was a business plan that was pitched to potential investors. Based on these initial explorations I entered the PhD programme at The Bartlett. In the last three years I conducted a series of media artistic installations for festivals around the world. All these experiences feed into this PhD thesis. The largest installation, which will be part of this research is called the Sentiment Cocoon. The Sentiment Cocoon was the winner of a competition organised by structural engineering company Arup Ltd. in London. In the following it was realised under an extremely short construction design phase (10 weeks). As a result and worth sharing, there is much we have learnt as young practitioners ranging from winning a competition to actually planning and realising a design as well as making this part of research.

% \end{preface}




\begin{acknowledgements}
Thank you. Thank you! Thank you!!
\end{acknowledgements}

\setcounter{tocdepth}{2} 
% Setting this higher means you get contents entries for
%  more minor section headers.

\tableofcontents
\listoffigures
\listoftables

