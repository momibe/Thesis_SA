\maketitle
\makedeclaration
%\makeindex

\begin{flushleft}

\textit{The timeless task of architecture is to create embodied and lived existential metaphors that concretise and structure our being in the world. Architecture reflects, materialises and eternalises ideas and images of ideal life. Buildings and towns enable us to structure, understand and remember the shapeless flow of reality and, ultimately, to recognise and remember who we are. Architecture enables us to perceive and understand the dialectics of permanence and change, to settle ourselves in the world, and to place ourselves in the continuum of culture and time.}\newline
Juhani Pallasmaa 
\end{flushleft}




\begin{abstract} % 300 word limit
\begin{onehalfspace}{
Little is known about the effect novel media technologies actually have on our everyday life and particularly on our notion of architecture. 
New practices of technologically mediated social relations emerge, which are inherently spatial.
Within the built environment large dynamic displays increasingly become a material to support dynamic and reactive architectural concepts and designs. 
But how can we design for interactions with large electronic surfaces in public space?
I understand interactions as shared encounters between humans or humans and non-humans such as conversations between strangers at a bus stop or a traveller instructing a ticket machine. 
Mediated interactions involve interfaces to assist or enable those mutual animations. 
Considering interfaces as time and space related processes that involve media technologies rather than mere objects, I show that interfaces require surfaces to facilitate mediated interactions. 
Our urban mediascapes host these surfaces, which either can be individual mobile devices or situated large electronic surfaces such as urban screens or media facades. 
I focus on the later and coin the term Media Architectural Interfaces (MAI) to argue for the unity of interfaces, surfaces and the specific context of public space when designing for interactions. 
To examine this inquiry I went on a field trip into \textit{Sentiment Architectures}. 
This neologism, derived from the computational analysis of people's sentiments expressed through social media platforms, frames architectural structures that imply a spatial and temporal dimension while absorbing, reflecting or evoking human or non-human emotions.
This field trip is an applied research including the design, implementation and evaluation of several \textit{Sentiment Architectures} that are in line with the MAI framework and specifically examines social interactions in public space.
My findings suggest that designing for interactions with large electronic surfaces requires a holistic appreciation of the components of MAI and the social with a specific focus on the spatial context of the desired public space. 

}
\end{onehalfspace}
\newpage

\textbf{Preface}

In my previous life I have been trained as a maker and architect working for many years on a wide range of design projects in the industry.
In my new life I entered a world that went beyond the physical and I discovered digital technologies, interactive systems and mediated human behaviour. 
People from both of my lives asked me: why? what for?
I ve been considered neither fish nor fowl. I guess they meant working interdisciplinary. In the end I found out that my contribution is to bring the knowledge I gained in my second life back into my first life with the hope to have left some traces of architectural thinking in my second life.

This research commenced during my attendance in the MSc Adaptive Architecture and Computation programme at UCL The Bartlett School of Architecture in 2010/11. 
As part of the course module "Cities as Interface" run by Ava Fatah gen. Schieck I came up with the physical \textit{I like} button.
This novel tangible interface, that until then existed only in digital space, allowed people for the first time to comment on experiences in situ and share them with their remote friends on digital social networks in real-time \cite{Behrens_2011}. 
Back then I was trying to explore this innovative research not only from an academic perspective but also aimed to investigate how we could turn this research and its technology into a business. 
In this respect I was taking part in the UCL Advance "StartUp Summer", in collaboration with YouGov. 
In a team we developed a business plan for a venture that would implement feedback sensors in businesses to improve the business-consumer relationship. 
The result was a business plan that was pitched to potential investors. Based on these initial explorations I entered the PhD programme at The Bartlett. 
In the last three years I conducted a series of media artistic installations for festivals around the world. All these experiences feed into this PhD thesis. 
The largest installation, which will be part of this research is called the Sentiment Cocoon. The Sentiment Cocoon was the winner of a competition organised by structural engineering company Arup Ltd. in London. 
In the following it was realised under an extremely short construction design phase of only 10 weeks. 
As a result and worth sharing, there is much we have learnt as young practitioners ranging from winning a competition to actually planning and realising a design as well as making this part of research.

VEIV doc centre
\end{abstract}






\begin{acknowledgements}
\textit{Thank you. Thank you! Thank you!!!}

Bruce Nauman
\newpage

\listoftodos



\newpage

%EU funded Connecting Cities Network lead by Susa Pop
%Ars Electronica for showing our work at the prestigious Ars Electronica Festival in Linz
%Dan Hirschmann for his inspiration for creative use of technology and  initial RFID support
%Giles Rudkin for his insane view on our technology mediated social being.
%Christian Nold for challenging my enthusiasm in digital technologies.
% Tom Haynes and Martin Traunmüller for the pub crawling
% 31 Warham Road for giving me a home with lovely peeps
%Tonya Nelson and the Petrie Museum for her support and enthusiasm 
%YouGov and Morris for getting me excited about data
%UCL Advances and ... for letting me explore the business side of creative technologies
%I immensely benefited from the patient and reliable support of Dejan M and the UCL VEIV Centre 
%UCL Advances
%Elka and Vesselin for their xxx hospitality and loving support during the writting phase
%Ian Kurker (check spelling) for the Latex support

\textbf{Publications}

\begin{singlespace}{
The research presented in this dissertation has been partially published in the following publications. This includes some figures, tables and written parts:

\begin{itemize} 

\item Behrens, M. (2017) \textit{Sentiment Architectures as Vehicles for Participation}. Chapter in: Bullivant, L. (Ed.) \textit{4D Hyper-local: A Cultural Tool Kit for the Open Source City}. Wiley, London.

\item Behrens, M., Fatah, A. (2016). \textit{Design Space for Media Architectural Interface}. Chapter in: Pop, S., Toft, T., Calvillo, N., Wright, M. (Eds.) \textit{What Urban Media Art Can Do}. Berlin: Avedition.

\item Behrens, M., Berkes, C., Wohlgemuth, S. (Eds.) (2016) \textit{Sentiment Architectures – A Field Guide to Behaviour and Emotion in Time and Space.} botopress, Berlin.

\item Behrens, M., Fatah, A., Brumby, D. (2016). \textit{Designing Media Architectural Interfaces for Interactions in Urban Spaces.} Chapter in: Foth, M., Brynskov, M., Ojala, T. (Eds.) \textit{Citizen’s Right to the Digital City: Urban Interfaces, Activism, and Placemaking}. Singapore: Springer.
%\item Behrens, M. and Fatah, A. (2015). \textit{Framework for Media Architectural Interfaces for Interactions in Urban Space.} Poster in: Proceedings of the 10th International Space Syntax Symposium, 2015, London, UK.
%\item Behrens, M., Valkanova, N., Mavromichalis, K., Fatah, A. (2015). \textit{Exploring DIY Sentiment Interfaces in Mediated Urban Spaces.} Workshop at: MEDIACITY 5 International Conference and Exhibition 1st – 3rd May 2015 Plymouth, UK.

\item Valkanova, N., Claes, S., Behrens, M., Vande Moere, A. (2015). \textit{Information-Bombing: Confronting the Public to Civic Data.} In: Workshop on Beyond Personal Informatics: Designing for Experiences with Data, co-located with CHI’15.

%\item Behrens, M. (2014). \textit{Designing Interactive Media Architecture.} In: Non-Refereed Proceedings of the 2nd Media Architecture Biennale Conference: World Cities 2014, Aarhus, Denmark, MAB14.
\item Behrens, M., Valkanova, N., Fatah, A., Brumby, D. (2014). \textit{Smart Citizen Sentiment Dashboard: A Case Study Into Media Architectural Interfaces.} In: Proceedings, PerDis’14: The International Symposium on Pervasive Display, Copenhagen, Denmark. ACM library.
%\item Fatah, A., Schnädelbach, H., Motta, W., Behrens, M., North, S., Lei Ye, Kostopoulou, E. (2014). \textit{Screens in the Wild: Exploring the Potential of Networked Urban Screens for Communities and Culture.} In:Proceedings, PerDis’14: The International Symposium on Pervasive Display, Copenhagen, Denmark. ACM library.
%\item Memarovic, N., Fatah, A., Kostopoulou, E., Behrens, M., Al-Sayed K. (2014). \textit{Attention, an Interactive Display is Running! Integrating Interactive Public Display Within Urban Dis(At)Tractors.} In: Screencity Journal Special Issue 4, 2014.

%\item Fatah, A., Al-Sayed, K., Kostopoulou, E., Behrens, M., Motta, W. (2013). \textit{Networked architectural interfaces: Exploring the effect of spatial configuration on urban screen placement.} In: Proceedings of the Ninth International Space Syntax Symposium Edited by Y O Kim, H T Park and K W Seo, Seoul: Sejong University, 2013.
\item Behrens, M. (2013). \textit{Living Light Lab: Exploring Instant Feedback in Mediated Urban Space.} UbiComp ’13 Adjunct, September 08 – 12 2013, Zurich, Switzerland.
%\item Behrens, M., Memarovic, N., Traunmueller, M., Rudkin, G., Fatah, A. (2013). \textit{ExS 2.0: Exploring Urban Spaces in the Web 2.0 Era.} Workshop at: 6th International Conference on Communities and Technologies
\item Behrens, M., Fatah gen. Schieck, A, Kostopoulou, E., North, S., Motta, W., Ye, L., Schnädelbach, H. (2013). \textit{Exploring the effect of spatial layout on mediated urban interactions.} In: (Proceedings) , PerDis’13: The International Symposium on Pervasive Display. ACM library
\item Behrens, M., Fatah. A. (2013). \textit{Exploring spatial configurations and actors, spectators and passers-by role in mediated public spaces.} CHI’13: Experiencing Interactivity in Public Spaces EIPS, Paris, France. (workshop position paper)
%\item Al-Sayed, K., Fatah g0en Schieck, A., Kostopoulou, E., Behrens, M.,  Motta, W. (2013). \textit{Networked architectural interfaces: Exploring the effect of spatial configuration on urban screen placement.} Proceedings of the Ninth International Space Syntax Symposium. Seoul: Sejong University, 2013.
%\item Memarovic, N., Fatah gen Schieck, A, E. Kostopoulou, M. Behrens, and M. Traunmueller (2013). \textit{Moment Machine: Opportunities and Challenges of Posting Situated Snapshots onto Networked Public Displays.} In Proceedings of the 14th IFIP TC13 international conference on Human-computer interaction (INTERACT 2013), Marco Winckler (Ed.), Vol. Part x. Springer-Verlag, Berlin, Heidelberg.
%\item North, S., Schnädelbach, H., Fatah gen. Schieck, A., Motta, W., Ye, L., Behrens, M.,  Kostopoulou, E. (2013) \textit{Tension Space Analysis: Exploring Community Requirements for Networked Urban Screens.} In Proceedings of the 14th IFIP TC13 international conference on Human-computer interaction (INTERACT 2013), Marco Winckler (Ed.), Vol. Part x. Springer-Verlag, Berlin, Heidelberg.
%\item Motta, W., Fatah gen. Schieck, A., Schnädelbach, H., Kostopoulou, E., Behrens, M.,  North, S., Ye, L. (2013) \textit{Considering Communities, Diversity and the Production of Locality in the Design of Networked Urban Screens.} In Proceedings of the 14th IFIP TC13 international conference on Human-computer interaction (INTERACT 2013), Marco Winckler (Ed.), Vol. Part x. Springer-Verlag, Berlin, Heidelberg.

\item Behrens, M. (2011). \textit{Swipe ‘I like’: location based digital narrative through embedding the ‘Like’ button in the real world.} Presented at: 5th International Conference on Communities and Technologies – Digital Cities 7.
\item Behrens, M. (2011). \textit{Can we enhance the engagement of online and real world communities with museum content through embedding a location‐based ‘I like’ button?} Thesis for MSc Adaptive Architecture and Computation, Bartlett School of Graduate Studies, University College London.



\end{itemize}
}
\end{singlespace}

\newpage

\textbf{Projects}
\label{sec:projects}

\begin{singlespace}{
The applied research in this dissertation is founded on a series of design projects I have conducted in collaboration with artists, designers and creative technologists since 2011. 

\begin{labeling}{projects}

\item [\textbf{2016}] 
\begin {itemize} 
%\footnotesize
\item The latest version of the \textit{Sentiment Cocoon} is a standalone structure that was produced for the Vivid Light Festival 2016 in Sydney, Australia. This project was implemented in collaboration with lighting designer Konstantinos Mavromichalis. 
\end{itemize}

\item [\textbf{2015}] 
\begin {itemize} 
%\footnotesize
\item In collaboration with lighting designer Konstantinos Mavromichalis the \textit{Sentiment Cocoon} won the No.8@Arup 2015 competition. Subsequently the cocoon was built at ARUP Ltd. headquarters in London, UK. 
\end{itemize}

\item [\textbf{2014}] 
\begin {itemize} 
%\footnotesize
\item As part of the EU funded Connecting Cities Network the \textit{Smart Citizen Sentiment Dashboard} was on display during the Staro Riga Light Festival in Riga, Latvia. The set up consisted of a transportable public display and was positioned close to the main train station. This project has been delivered with support of interaction designer Nina Valkanova. 

\item Again the Connecting Cities Network invited the \textit{Smart Citizen Sentiment Dashboard} to the Ars Electronica Festival in Linz, Austria, where we connected the dashboard to the media facade of the Ars Electronica building. Collaborators in this projects were interaction designer Nina Valkanova and creative technologist Martin Nadal.
\end{itemize}

\item [\textbf{2013}] 
\begin {itemize} 
%\footnotesize
\item The first version of the \textit{Smart Citizen Sentiment Dashboard} has been invited to the Viva Cidade Festival in Sao Paulo, Brazil. The interface was connected to a large retrofitted media facade at the FIESP building in Avenida Paulista. This project has been realised in collaboration with interaction designer Nina Valkanova. 
\end{itemize}

\item [\textbf{2012}] 
\begin {itemize} 
%\footnotesize
\item The next version of the \textit{Swipe I like} project got a stage during the anniversary of the UCL Centre for Virtual Environments, Imaging and Visualisation (VEIV) where we designed a large DIY display for the interface in the UCL Main Quadrangle.  
\end{itemize}

\item [\textbf{2011}] 
\begin {itemize} 
%\footnotesize
\item The \textit{Swipe I like} project formed the case study for my master thesis at UCL The Bartlett. The implementation of a series of physical I like buttons at the UCL Petrie Museum provided the foundation for this dissertation.   
\end{itemize}

\end{labeling}

}
\end{singlespace}




\end{acknowledgements}


\setcounter{tocdepth}{5} 
% Setting this higher means you get contents entries for
%  more minor section headers.

\tableofcontents
\listoffigures
\listoftables

