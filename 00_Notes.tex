
Joerg Mueller uses the following structure in his paper:

1) observed behaviour ... 2) Design recommendation ...

\section{Comments from Ava on draft: 12.10.16}

\begin{itemize}
\item the term temporary in the title would suggest that all media architecture I refer to is of a temporary nature - needs to be clarified
\item a thesis is a narrative and not subdivided in subdevisions
\item Ava finds the term technical requirements too confusing in this context as it is too engineering, she suggests to use terms like components or elements
\item limitations, constraints and restrictions are driver for design
\item SPATIAL COMPONENTS form the main contribution in this thesis
\item the thesis is about the argument and not the projects
\item Ava likes the title: A field trip into Sentiment Architectures
\item Each thesis has a character - which character has mine?
\item the Background establishes the foundation of a thesis ... everything that is there appears again somewhere else in the thesis 
\item GIVE A CLEAR definition of what I define as INTERACTIONS and INTERFACES
\item Most important is MY VIEW on the topic 
\item Taxonomy goes into the discussion 
\item Findings that are currently in the discussion go at the end of each project
\item keep the hierarchical order right: FIELD TRIP - STUDY - PROJECT
\item each CASE STUDY needs to be standalone - meaning: Introduction, methodology, case study, findings, discussion/conclusion
\item in the Appendix there should be a section for media mentions - in there are awards, competitions ...
\item use HAWARD style guide for citation
\item a part of the methodology is to have INFORMANTS
\item regarding VIVID: What can I generate out of it for my PhD?
\item ARUP robot issue: I can talk about non-human behaviour in the introduction and background and then get it into the case study as pre-phase and post-phase (don t mention the robotic aspects - they d only open up a new topic)
\item performative aspects of the interfaces
\item Steve Ridge is the person to submit the CRS application to 
\item Venturi, Learning from Las Vegas, Cedric Price, Fun Palace, Centre Pompidou}
\end{itemize}




\section{Forming a research question:}

\begin{itemize}
\item In this thesis I outline how interaction design may engage in the exploration and understandings the material and mediation of new interface technologies.
\item I demonstrate and analyse how interaction design research might address the inter-related issues of invisibility, seamlessness and materiality that have become central issues in the design of contemporary interfaces.
\item It investigates a particular emerging interface technology called Radio Frequency Identification or RFID, that was originally developed from radar systems and is now used to identify objects at a distance through small embedded tags and readers.
\item These issues are analysed and developed through three exploratory and inventive approaches
\item The central question I address is:
\item In addressing these questions, the thesis takes up the challenges for interaction design research in the exploration and communication of new interactional materials.
\item To respond to these questions by design and through analysis, the thesis focuses on three kinds of interconnected design research approaches that involve culture, material and communication detailed in Chapter 2 and 3.
\item These practices are presented and analysed through a multimediational mixture of text, photographs and films.
\item In the next section, I outline the starting point of the research, and why it became important to start to address the invisible materials of interaction design through media.

General formulations:

\item In essence, ...
\item The argument of the thesis is that ...


\end{itemize}
