\chapter{Introduction}

\label{chapterlabel1}

\section*{Chapter summary}

In this chapter I will introduce my research, which will lead to the main research question and the subsequent question that will assist to answer the main research question. Further an outline of the thesis will be conducted. This chapter then concludes with the overall contribution the reader can expect from reading this thesis.\newpage

\section{Motivation}

Little is know about the effect digital technologies have on social interactions in urban space and eventually on our notion of architecture. Cities and their inhabitants are currently familiarising themselves with the
socio-technological developments of the emerging digital age \cite{Hemment2013}. Simultaneously, computer technologies increasingly affect architecture in the way that buildings turn into computer interfaces \cite{Mitchell1996}. Today,
information and communication technologies (ICTs) may be considered to be of
structural, cultural and formal nature \cite{Saggio2013}. Human behaviour is increasingly
structured amongst others through information displays and ubiquitous
mobile devices characterise our everyday culture. At the same time, architecture
is adjusting to the requirements of novel ICT such as large digital displays
attached to building facades. For instance, the flashy New York Times Square is
an obvious example of where digital media technology has been materialised in
architectural form. In essence, the urban environment today can be considered as a
system that integrates the human, architecture and ubiquitous computing technologies
\cite{Fatah2006}. Within this system are large programmable electronic
displays such as urban screens, media facades or media architecture.
These novel ''digital" surfaces are gradually turning buildings into responsive media facades, which by using display technologies as an architectural material, radically alter the formal and informational character of the building \cite{Fatah2006}\cite{Ebsen2013}.
Whereas the notion of architectural facades has always been of informational value, although static, in the way that a facade was considered a social interface connecting the bourgeoisie (private space) with the surrounding city (public space) \cite{Neumeyer2002}, the advent of ICTs has allowed for new types of facades to emerge: information is literally not
set in stone anymore \cite{Haeusler2009}. Digital information also allows for the anytime
exchange of information, such as the flickering advertisement billboards in
an inner city, whilst at the same time, the nature of anytime and anywhere information
access is changing the relationship between the public and private space.
This has led to the development of novel forms of interactions between humans
and humans, humans and computers, and humans and architectures. And although
Manovich clearly states that the nature of human–computer interactions (HCIs) is
interactive per se \cite{Manovich2001}, architectural surfaces are by far not.

Having this in mind, one may ask how to go about designing for interactions
with large programmable electronic displays? Part of the challenge is that there
are currently only a handful of large interactive surfaces in existence, and so there
is much to learn from each attempt to deploy interactive systems. 

Hence, the work outlined in my PhD thesis contributes to HCI as well as architectural research by juxtaposing
existing interaction frameworks that are concerned with the awareness of
spaces mediated through ICT, participants and their actions within these spaces as well as the properties of these spaces, which frame these interactions, and are surrounded by the physical built environment. I introduce the notion of media
architectural interfaces (MAI), which is then supported through the description of
the design, deployment and evaluation of a series of design studies: 
VEIV London and
SCSD Sao Paulo. 

Finally, we discuss the multilayered interaction frameworks with regard to the conducted design studies and summarise the relevant communalities of these design studies in a taxonomy. The aim of this categorisation is
to provide design implications for future MAI projects. Ultimately, this may support
the design and development of novel and sustainable interactive systems in
the domains of urban screens, media facades, and media architecture.


%\subsubsection* {From mobile DIY displays to media architecture }  
 
\section{Key concepts}

\subsubsection* {From urban computing to media architecture}

Architectural research, such as Space Syntax research, has dealt with the relationship
between architecture and human behaviour. The city is considered as an
arrangement of architectural layouts that are defined through their relationships
between physical space and social life reflected in movement patterns and activities
of its inhabitants \cite{Hillier1984}. Space Syntax aims to analyse
the spatial morphology of cities through researching the ‘relation of space to society,
[which] is mediated by spatial configuration. Spatial configuration proposes
a theory in which we find pattern effects from space to people and from people
to space’ \cite{Hillier1998}. A methodological tool-set provided by Space
Syntax facilitates the systematic study through spatial analysis and empirical
observations of human behaviour such as pedestrian movement or social encounter
in the urban realm \cite{AlSayed2013}. In this context, social encounters
can be seen as (un-)planned gatherings amongst strangers or people who know
each other. Architectural research has defined ''shared encounters" as mostly context
aware, and hence, the type of encounter stage (e.g. bus stop or museum) and
its information context impacts the kind of shared encounters. Encounter stages
can be defined as public spaces ''on which people negotiate boundaries of a social
and cultural nature" \cite{Fatah2009}. For example, at bus
stops, social chance encounter happen when people ask for directions or start conversations
about the delayed schedule. One objective of the research presented in
this paper is to focus on the particular spatial properties, which physical encounter
stages require in order to support shared encounters.
Since the advent of ubiquitous computing \cite{Weiser1991}, and its application
in urban space in the form of urban computing \cite{Kindberg2007}, the built environment incorporates architecture and ubiquitous computing technologies.
In our work, we focus in particular on large electronic displays in public spaces
(i.e. public displays), media facades and media architecture and their potential for
mediating social interactions. Consequently, architecture and HCI are concerned
with the design, deployment and evaluation of digital media technologies and their
effect on social interactions in urban spaces. Seen from an architectural perspective,
urban screens merged into media facades and became part of the architectural
repertoire. Accordingly, media facades are visually animated architectural
surfaces, such as dynamic light facades \cite{Virilio1991} \cite{Fatah2006}
\cite{McQuire2009}.
HCI research, on the other hand, is concerned with interactions in general
and increasingly with social interactions, which are mediated by computer technologies.
More recently, a branch of HCI began to study social interactions and
human–computer interactions in the real world \cite{Rogers2011}. Inevitably, this led
to a discussion about space within this field. Since then HCI research has conducted
extensive research in understanding technology-mediated human behaviour
and social interaction in public space \cite{Fischer2012} \cite{Akpan2013}.



\section{Problem statement}

\begin{itemize} 

  \item 
  
  \item 
  
  \item 
  
  
\end{itemize}

 
\section{Research questions}

 
The central question I address is:
\begin{itemize}
  \item How are technology mediated interactions changing/challenging our notion of architecture?
  \end{itemize}
  
\noindent Subsequently this question leads to a number of related issues:
  
 \begin{itemize} 
  \item How did human behaviour change in architecture since the digital age? (anthropometric vs. human behaviour)
  \item How to go about designing for interactions with large programmable electronic displays?
  \item How may large programmable electronic display connected to media architectural interfaces challenge our notion of architecture?
\end{itemize}


Little is know so far about the effect digital technologies have on social interactions in urban space. Therefore this research is of exploratory nature and applies a cross-disciplinary methodology.  


\section{Chapter outline summary}

different output formats (dvi, pdf)

\section{Contribution}

\subsubsection*{Interactive media facades} Summarising state of the art of interactive media facades and content
\subsubsection*{Design processes} Describing the design process and identifying challenges through own projects 
\subsubsection*{Interdisciplinary collaboration} Contributing to the notion of interdisciplinary collaboration between design, technology and research 
\subsubsection*{Guidance} Guidance for architects and urban designers aiming to use media architecture for enriching urban spaces




