\chapter{Cases}
\label{chapterlabel5}

\section*{Chapter summary}

This chapter will present the conducted case studies in support of the research questions. I will introduce one pilot study, which originally emerged during my MSc studies. Followed by five in depth studies.  All studies are based on the same digital technology to mediate human interaction, but differ significantly in In describing these case studies, I explore the deployment of each MAI and highlight the implications for designing interactions. The focus is on mediated interactions in the vicinity of the deployed MAI. Both case studies differ in the architectural scale and the nature of the urban space (i.e. pedestrianised courtyard vs. congested high street). \newpage

\section{Swipe I like}

\subsection{The Petry Museum}


\subsection{Discussion}



\section{VEIV London}

\subsection{The Building Projection Party}

This study \cite{Behrens2013} (Behrens 2013) is a follow-up on the early implementation we conducted using a TUI of the digital ‘I like’ button (Fig. 3), situated in the physical space within a museum context \cite{Behrens2011} \cite{Behrens2011} (Behrens 2011a, b). The findings suggested that visitors are keen to leave their feedback and share it with others (in this case through a simple swipe card interface connecting to Twitter and Facebook).
Analysis of the submitted data discovered movement and encounter patterns of
visitors interacting with the installed devices and the observations of interactions
revealed various forms of human behaviour such as strangers discussing the purpose
of the new interfaces. In other words, the space around the tangible device
turned into an encounter stage that did not exist at this place before. However, it
became clear that a real-time visualisation or representation of the gathered data
shown to the audience as an immediate feedback of their actions was desired by
many participants but missing in this initial implementation.
Building on these initial findings, the purpose of the VEIV London study was
to: first, explore whether a feedback device that works in a condensed indoor
public space can also be deployed in an outdoor public setting full of urban distractions,
and second, add a dynamic display to the device to fulfil the need of
providing the user with an immediate visual feedback. Consequently, we ask what
kind of social interactions and dynamic relations of human behaviours take place
in an urban setting when mediated by this interactive installation.
The study took place during the UCL Virtual Environments, Imaging and
Visualisation doctoral centre (VEIV) festival, 2013, where we set up the first
prototype of the LED display at the UCL main campus. To explore interactivity,
we connected the light installation to the binary feedback device based on Radio
Frequency IDentification (RFID) technology, which was used in the indoor public
space of the museum \cite{Behrens2011} (Behrens 2011b). During the festival, participants were able to leave their feedback in a playful way through swiping their travel cards (Oyster)
or UCL access cards over the thumbs-up or thumbs-down icons on the card reader.
Basically, the simple but effective question was whether visitors like or don’t like
the VEIV Centre. When swiping across the thumbs-up icon, the light installation
turned the VEIV logo into a warm orange, whilst the thumbs-down icon turned the
logo into a cool blue (Fig. 4).
The LED display consists of three low-resolution displays following the principle
of 16 digit number displays, which allow the projection of numbers and letters.
In total, 48 RGB 24V LED light strips are connected to micro-controllers, which
are addressable through the DMX protocol. A decoder transfers the DMX signals
through USB wire to a laptop. The visualisations are programmed with processing.
The installation was running between 8 and 10 pm on a summer day. During
the first hour, the daylight reduced the light distribution of the LED light, whereas
with the approaching darkness the LEDs were colouring the surrounding buildings
in either orange or blue ambient light.
Throughout the event, we took pictures, notes and informally talked to people
joining the event. Overall, participants enjoyed using their Oyster Cards or UCL
access cards to change the colour of the low-resolution display. The fact that they
were actually rating the event was less important than the playfulness of changing
the colours.
We observed people sitting on the lawn suddenly walking up to change the colour
from the apparently uncomfortable blue to the warm orange. They obviously
felt more comfortable with the cosy orange light than with the cold blue light.
A kind of campfire atmosphere was created where people were sitting around a
source of pleasant light that illuminates the faces of others as well as putting the
surrounding facades of the classic campus building into an orange-red shade.
The lawn in front of the installation was the preferred seating area and the TUI
(i.e. ‘I like’ device) created a stage for social encounters. However, after sunset,
people moved away from the immediate space around the display as the LEDs
became too bright. Interestingly, people neither sat nor stood in between the tangible
interface and the LED display nor behind the LED display.
The DIY and low-resolution media display worked out very well considering
the relatively low effort that was put into the design and making of the low-resolution
screen and the visual impact of illuminating a courtyard during a large public
event was huge. The fact that the lightweight frames are easily movable makes this
project easily adaptable to other locations.


\section{SCSD Riga}

How do the citizens of Riga actually feel about their city? What do they think about urban topics that concern the development, public transport, environment, safety and culture of their city? As part of the Participatory City, initiated by the Connecting Cities Network, Staro Riga Festival invited us to show our work. 

The project called Smart Citizen Sentiment Dashboard (SCSD) is an interactive and participatory installation that lets citizens engage with and comment on urgent urban challenges concerning the city of Riga. The SCSD project aims to translate citizen feedback into a visual language, which is displayed on a large LED display. The media facade and its surrounding turn into a stage for social encounter where citizens meet and urban challenges can be discussed.

\subsection {Staro Riga Festival}

During the Staro Riga Festival the installation was set up on a busy pedestrian crossing at the intersection of Marijas Iela and Satekles Iela, which is close to the Riga main train station. The visualization was displayed on a large mobile screen facing towards Satekles Iela, and the sentiment dashboard was set up in front. The installation ran daily in between 6pm and 11pm starting Friday, 14 November 2014 and finished at the National Independence Day on Tuesday, 18 November. During the five days of deployment, in which the installation was running for five hours each evening, the interactive system tracked approximately 1600 interactions.

\subsubsection{The Dashboard}

The motivation was to develop, design and deploy a situated and tangible system that mediates collaborative interactions in public spaces whilst focusing on accessibility and affordance. In other words, the interface should be understandable and easy to use for people. The employed technology makes use of existing ‘Radio frequency identification’ (RFID) as known from smart card technology such as the e-talon in Riga. We build on the widely spread use of these unique ID tags for payless travel purposes, as a large proportion of citizens of Riga carries an e-talon in their pocket. Consequently the use of these cards is a recurring embodied interaction in the smart city. At the same time every interaction is uniquely identifiable and therefore traceable. Our aim was to allow people to use their ID tags beyond technical purposes and express their mood and opinion about specific issues in the technology mediated urban realm. Hence the Smart Citizen Sentiment Dashboard (SCSD) enables participants to express their mood about urgent urban challenges in the city of Riga. The challenges were defined as follows: 1) ATTĪSTĪBA (development), 2) TRANSPORTS (transport), 3) VIDE (environment), 4) DROŠĪBA (safety) and 5) KULTŪRA (culture). By switching a knob on the device participants are able to choose one of the aforementioned categories. By swiping their RFID token (i.e. e-talon) across one of the two emoticons (happy or sad) their mood was transmitted onto the LED display. The SCSD affords three folded interactions: 1) switching: 5 categories can be selected through a rotary switch; 2) swiping: after choosing the category, the electronic ID card needs to be swiped over one of the three mood states (happy, indifferent, sad); 3) pushing: finally a simple push-button (Red Button) allows users to view the overall feedback of all collected moods.

\subsubsection{The Visualisation}

We chose a visualisation technique that combines the “seriousness” of the topic with the more accessible style of popular info-graphics. The visualisation consists of an abstract sunburst representation, of which each burst corresponds to the sentiment of an individual participant towards the currently selected urban challenge. Each urban challenge is encoded by a different colour and an icon representation. Upon switching the rotary knob, the sunburst visualisation corresponding to the specific urban challenge, and coloured accordingly appears on the facade.

The sentiment ‘value’ for each participant (happy, un-happy) is graphically encoded through the length of the corresponding burst: the longest burst represents a positive sentiment towards the urban challenge at hand, while the shortest corresponds to a negative statement. Our choice for this circular visualisation technique was also motivated by its scalability, which allows for an arbitrary number of people to participate and be visually represented. We considered this flexibility a desirable feature in the context of urban environments, often characterised by highly variable and open-ended, and unpredictable flux of people and interactions.

\subsubsection*{Animations}

The integration of dynamic visual cues can make visualisations richer, more vivid and therefore easier to understand. Accordingly, our visualisation shows a dynamically animated circle over the sunbursts in order to convey the average participants’ sentiment for the given urban challenge. Each new burst from a participant visually appears with a smooth animation and bouncing effect, to highlight the recording of fresh data. A new entry is displayed in a white color to unambiguously distinct it from the rest of the graphical representation. Shortly after it is smoothly taken over by the color of its respective urban challenge.

\subsubsection*{The Red Button}

In order to provide citizens with an overview of previously submitted sentiments, and a more interactive approach to exploring the installation, we integrated a ‘Red Button’ at the bottom of the interface. When pushing this button, a dynamic visualisation of the average feedback for all available urban challenges is represented on the facade. As mentioned above, each urban challenge is represented by its corresponding colour, and occupies a different part of the circular shape proportionally to the relative participation rate of the according challenge. We aimed to create a simple, playful, yet meaningful approach to enable citizens and participants alike to make a deeper sense of the installation, and the underlying participation results: people can gain insight about which urban challenge is most attractive to vote for, and what is the average sentiment about it of fellow citizens. This, beyond being an overview, the heart visualisation symbolises the overall ‘sentiment’ of the city towards its urban challenges.

\subsection{Riga’s Sentiments}

The installation was running for five days, in total for 25 hours. During this time the database logged about 1600 interactions. These interactions cannot automatically be counted as single votes by individuals as many participants were exploring the installation in a playful manner through swiping their e-talon more often. In a next step with an elaborate data cleaning process this behaviour can be removed from the data set. However, we can already identify a clear tendency that shows that participants overall voted positive, as 75 percent of the interactions were happy. If we look more closely into the different urban challenges we can reveal more telling insights of how the citizens of Riga think about their city. ATTĪSTĪBA (development) was the category, which attracted with 356 interactions the least attention of participants. Still 76 percent (270 counts) of the recorded data was positive. As development maybe interpreted in many ways participants could have felt irritated and therefore may have decided not to vote in this category. TRANSPORTS (transport) and DROŠĪBA (safety) were the most controversial topics, which is by the way in line with other cities where the installation was running before. Transport attracted 379 interactions and safety 434 interactions of which both topics were 65 percent happy and 35 percent un-happy. As there were more interactions tracked for the category of safety one may argue that this is due to the higher importance of this topic to participants. VIDE (environment) attracted 447 interactions of which 78 percent were positive and 22 percent negative. KULTŪRA (culture) has been the topic most participants wanted to express their sentiments about. The system logged 518 interactions of which 441 (85 percent) were positive and only 77 (15 percent) negative. Culture is therefore not only the most frequently used urban topic but also the topic participants are the happiest about. One may argue that this result might be due to the fact that participants were influenced by the many cultural events that were going on at this time during the Staro Riga Festival.

\subsubsection{Observations}

Overall it is to say that the logged interactions by the system were on average 75 percent positive as well as none of the categories reached a majority of un-happy interactions. This is a result we did not find in any previous deployment and therefore lets us assume that Riga is a city where people seem to be quite content with the addressed urban challenges or tend to see things in a more positive way.

Our observations revealed three different participation patterns:

The serious behaviour

A participant submits exactly one sentiment for each of the explored categories. This pattern would reflect how we expected the interaction mechanism to work – i.e. a person would explore the categories by rotating the knob and would submit one sentiment for a specific preference.

The repetitive behaviour

This was the most frequently observed participation pattern. The participant has submitted the same sentiment (same preference for a certain category) several times within the considered time range. The occurrence of this pattern can be explained with our frequent observation of participants holding their card over the RFID reader (for a certain preference) for several seconds. Thus the system registers several submissions (although our system had restricted votes not to be registered within 5 seconds after each given participation). This behaviour might be due to a usability flaw of our installation – the participating person did not realize the effect of her participation in the visualization, hence tried several times. Another explanation might be the manifestation of a particular sentiment towards one urban challenge: by holding the card over the reader, the user might have wanted to reassure herself that her opinion would be registered by the system.

The playful behaviour

There were many occurrences of this behaviour during the deployment of the installation. The participant has submitted several different preferences for the same category within the considered period of time. This might indicate that s/he did not really want to express an opinion, but rather explored how the installation and the visualization work. While we cannot account for representative polling results, the findings indicate the installation fulfilled its intentions as an urban feedback platform, where people engage meaningfully with locally relevant topics. In the future it would be exciting to deploy several citizen sentiment dashboards permanently across the city as well as working closer with city authorities and local communities. This might open a fruitful dialog in between citizens and stakeholders of Riga.

\subsection{Discussion}



\section{SCSD Sao Paulo}

\subsection{Viva Cidade Festival}

The Smart Citizen Sentiment Dashboard (SCSD) is an interactive participatory
installation that lets citizens engage with, and comment on, urban challenges in
their cities. Through a tangible interface connected to a media facade, passersby
and participants on-site can submit their sentiments and simultaneously see
the effect of their actions projected onto the facade. The tangible urban interaction
device allows for an intuitive and accessible, yet identifiable and public way
of expressing ones view. The project aims to create an open, aesthetic dialogue
about urban challenges and invites citizen to engage, by playfully allowing them
to express their opinion and share and compare their views in the physical built
environment.
Our motivation was to design, develop and deploy a situated system that mediates
collaborative interactions in public spaces, whilst focusing on accessibility
and affordance. In other words, the interface should be understandable and easy
to use for people. Based on the feedback of the previous pilot study, described in
the previous section, we deployed and tested-out the TUI in various occasions and
improved the system iteratively. Similar to the VEIV London study, the employed
technology makes use of existing RFID technology as known from smart card
technology. We build on the widely spread use of these unique ID tags for payless
travel purposes or building access, as a large proportion of city dwellers carries
a RFID tag in their pocket (i.e. the Bilhette Unico is the transport card in
Sao Paulo). Consequently, the use of these cards is a recurring embodied interaction
in the technologically augmented city. At the same time, every interaction is
uniquely identifiable and therefore traceable. Our aim was to allow people to use
their ID tags beyond technical purposes and express their mood and opinion about
specific issues in the technology-mediated urban realm. Hence, the Smart Citizen
Sentiment Dashboard (SCSD) (Fig. 5, left) enables participants to express their
mood about urgent urban challenges in the city of Sao Paulo (Behrens et al. 2014).
Up front, we were running four design ethnographic workshops amongst
various social groups in Sao Paulo with the aim to learn about citizens’ urban
challenges. As a result of the collaboration, five categories were established:
(1) environment, (2) mobility, (3) security, (4) public space and (5) housing.
By switching a knob on the device, participants are able to choose one of the
aforementioned categories. By swiping their RFID token (i.e. Bilhette Unico)
across one of the three emotions (happy, indifferent and sad), their mood was
transmitted on to the media facade (i.e. display) (Fig. 3).
The mood expressed by the user (i.e. happy, indifferent or sad) is then projected
onto a huge LED media facade (i.e. display), which has been retrofitted in the
existing honeycomb facade of the pyramidal FIESP building.
The media facade is divided into three parts, which are situated on three different
sides of the facade. The biggest and main display faces to the opposite side of
the street, whereas the two smaller screens are directed to display to both directions
of Avenida Paulista. The threefold low-resolution LED facade is formed of
a network of approximately 26,000 LED clusters (pixels) embedded in 3700-m2
metal structure that covers the pyramidal FIESP building. The grid is approximately
13 × 13 cm. Each pixel consists of a module of four LEDs: 2 × R, 1 × G,
1 × B the luminous intensity is 4.5 cd/module.
For the display, we chose a visualisation technique that combines the ‘seriousness’
of the topic with the more accessible style of popular info-graphics. The visualisation
comprises of an abstract sunburst representation, of which each burst
corresponds to the sentiment of an individual participant towards the currently
selected urban challenge. Each urban challenge is encoded by a different colour
and an icon representation. Upon switching the rotary knob (Fig. 6), the sunburst
visualisation corresponding to the specific urban challenge and coloured accordingly
appears on the facade. The sentiment ‘value’ for each participant (happy,
indifferent, sad) is graphically encoded through the length of the corresponding
burst: the longest burst represents a positive sentiment towards the urban challenge
at hand, whilst the shortest corresponds to a negative statement. Our choice for
this circular visualisation technique was also motivated by its scalability, which
allows for an arbitrary number of people to participate and be visually represented.
We considered this flexibility a desirable feature in the context of urban environments,
often characterised by highly variable and open-ended, and unpredictable
flux of people and interactions. The integration of dynamic visual cues can make
visualisations richer, more vivid and therefore easier to understand. Accordingly,
our visualisation shows a dynamically animated circle over the sunbursts in order
to convey the average participants’ sentiment for the given urban challenge. Each
new burst from a participant visually appears with a smooth animation and bouncing
effect, to highlight the recording of fresh data. A new entry is displayed in a
white colour to unambiguously distinct it from the rest of the graphical representation.
Shortly after, it is smoothly taken over by the colour of its respective urban
challenge. In order to provide citizens with an overview of previously submitted
sentiments, and with a more interactive approach to exploring the installation, we
integrated a ‘heart’ button at the bottom of the interface (Fig. 7). When pushing
this button, a dynamic visualisation of the average feedback for all available urban
challenges is represented on the facade. As mentioned above, each urban challenge
is represented by its corresponding colour and occupies a different part of the circular
shape proportionally to the relative participation rate of the according challenge.
We aimed to create a simple, playful, yet meaningful approach to enable
citizens and participants alike to make a deeper sense of the installation, and the
underlying participation results: people can gain insight about which urban challenge
is most attractive to vote for, and what is the average sentiment about it of
fellow citizens. This, beyond being an overview, the heart visualisation symbolises
the overall ‘sentiment’ of the city towards its urban challenges.

\subsection{Findings and discussion}

Our findings suggest that the implementation of a tangible interface not only supports
engagement indoor, such as in a museum context, but also outdoor during a
festive event and a media art festival. Yet, the actual TUI in an outdoor setting was
not immediately attracting participants. Instead, our observations suggest that the
strong visual presence of the display (i.e. low-resolution display or media facade)
appealed peoples’ attention first.
In the first study, VEIV London, the interactive system consisted of a simple
binary RFID swipe card interface. In the second study, conducted in Sao Paulo,
the shared user interface was again based on RFID technology but with additional
features such as a rotary knob and a push button. Initially, we decided to use RFID
technology instead of simple touch buttons due to the potential to identify the user.
We were able to track individual user behaviour such as returning users or misapplication,
which simple buttons would not allow. Despite these useful features,
we observed that the need for an individual swipe card generated a barrier that
was hindering some people to interact. The reasons for this may be manifold: in
London, the use of the transport card (i.e. Oyster Card) has a wider distribution
amongst all citizens. Public transport is generally considered to be a convenient
way of commuting and people feel secure using London buses or undergrounds.
Consequently, the usage of the Oyster Card appears to be more embodied than in
Sao Paulo, where private transportation is very common, mostly due to personal
security. Thus, Londoners more likely understand how to engage with the interface
and particularly liked the idea to use a smart card to give feedback about
services. However, we observed that the fact that not all people carrying a RFID
card in their pockets triggered unexpected social encounters during both studies.
Bystanders, curious to get involved, asked actors to borrow their cards, or families
sharing one card amongst each other. In addition, bystanders started debating
about the sense of having such interfaces in urban space or simply wanted to know
how the technology works.
During the SCSD Sao Paulo study, we looked closely at actors directly interacting
with the TUI. The most common behaviour actors performed started with
looking at the TUI, swiping their card or turning the rotary knob. After expressing
their feelings towards one of the categories at the swipe card interface, we saw that
participants would then frequently look up to the media facade to see what impact
their acting had on the visualisation. The fact that the visualisation was slightly
delayed (i.e. less than a second) was an advantage for the actor experience as it
left time for the body orientation towards the large media facade. Close bystanders
were behaving almost synchronic in order to understand the interaction modality
(Fig. 8).
At the same time, we made observations in the wider surrounding of the MAI
during the SCSD Sao Paulo deployment. We recognised MAI-related interactions
around the facade in the close interaction space as well as in the wider ambient
space. Although these observations were not rigorously conducted, we did notice a few recurring behaviours. In particular, we frequently saw people taking pictures
of the media facade with their mobile phones or taking pictures of each other in
front of the facade. These informal observations would suggest that people simply
liked the visual presence of the media facade’s visualisation and the heart icon it
used. Yet, the initial design aim was to develop an interactive system, which would
spark passers’-by engagement with the informational content and eventually
may trigger new social encounters and discussions about the matter. In the case
of VEIV London, participants were asked to express their opinion about the UCL
doctoral centre (i.e. like or don’t like), whilst during the SCSD Sao Paulo event
contestants could share their sentiments about urgent urban challenges in their
city (i.e. environment, mobility, security, public space or housing). Accordingly,
this raises the question regarding the informational character of the installation as
intended and the actual ambient perception of the visualisation by many people
in particular outside the direct interaction space in the close vicinity of the TUI.
Our findings here suggest that designers need to take the ambient perception of
MAI set ups into account when designing interactive systems in urban environments
(Fig. 9).
More specifically, we would like to relate our findings to the initially examined
interaction frameworks in Sect. 3, which dealt with the (1) awareness space,
(2) actor space, (3) action space and (4) physical space in the context of public
displays or a media facade. In the following, we will discuss these frameworks in
relation to the two design studies as outlined in the last section. We argue for the
consideration of these multilayered interaction spaces when designing interactive
systems in urban spaces.
Awareness space: In both studies (VEIV London and SCSD Sao Paulo),
the direct interaction activities took place in the immediate vicinity of the TUI,
whereas the focal and peripheral awareness spaces differed in their dimensions.
This is firstly due to the dissimilar size of the displays. The display used for the
VEIV London study was only 2 by 6 m (i.e. 12 m2) compared to the large media
façade for the SCSD Sao Paulo study with 3700 m2. At the same time, the difference
in distance between the user interface and the display was significant: 5 m
VEIV London compared to 33 m SCSD Sao Paulo. This seemed to have an impact
on people’s spatial awareness with regard to dynamic displays in the city.
Actor space: Compared to the dynamic stream of pedestrians in the SCSD Sao
Paulo study, the flow of people in the VEIV London study was rather calm, whilst
the event was highly attended. The audience hardly changed during the party, but
their actions altered; for instance, spectators turned into actors when walking up
to the TUI to change the colour of the display and returned to their friends. Others
simply enjoyed the presence of the colourful display or watched other guests interacting
with it. In contrast, the dwelling time of the audience during the Sao Paulo
event was much lower due to the high pass-through frequency of pedestrians on
their way to the underground station.
Action space: The flow of people’s actions during the VEIV London event was different
to SCSD Sao Paulo. Other than Michelis and Müller (2011) stated, the phases
through which people had to go before they could actually interact with the MAI
were not stringent in the VEIV London set up. People did not necessarily cross in the described order from (1) passing by to (2) viewing and reacting to (3) subtle interactions
to (4) direct interactions to (5) multiple interactions and to (6) follow-up actions.
Instead, we observed groups gathering on the lawn walking up to the TUI to collectively
explore the interactive system. In contrast to this, we identified human behaviour
as explained by Michelis and Müller (2011) during the SCSD Sao Paulo study.
Physical space: The spatial layout in the VEIV London case consisted of a
pedestrianised rectangular courtyard enclosed by the historical university buildings,
compared to the vibrant two-directional avenue in Sao Paulo, which was
lined by high-rise office towers. In the VEIV London set up, there was hardly
any gap space in between the TUI and the display (i.e. ca. 5 m), compared to the
congested main road in Sao Paulo, which divided the tangible interface from the
media facade (i.e. ca. 33 m). In addition, the comfort spaces, potential interaction
spaces and social interaction spaces varied greatly in both studies. The enclosed
courtyard and the positioning of the display within it released more of these spaces
than the congested pavement in Sao Paulo.
In summary, it appears that the affordance of each of the described spaces has a
strong impact on the presented studies and therefore need to be taken into account
when designing for such interactive systems in the built environment. With this in
mind, we return to our initial question: How to actually go about designing for an
interaction with MAI? To help answering this question, we now introduce a taxonomy
for the categorisation of MAI based on the triangular relationship between
TUI, display and setting. Within Fig. 10, we have plotted the relevant information
of both design studies against each other to gain a better understanding of the relation
between the individual properties.
In more detail, the properties of each constituent element will be described
below:
Interface: The characteristics for the participation level have been described by
Fritsch and Brynskov (2011) as (1) static, (2) dynamic, (3) reactive, (4) interactive,
(5) participatory and (6) communicative and recently have been extended by Caldwell and Foth (2014), who added the terms (7) performative and (8) controllable.
The characteristics of the interaction modalities originate from dimensions
summarised by Müller et al. (2010) and use the following interaction modalities:
(1) presence, (2) body position, (3) body posture, (4) facial expression, (5) gaze,
(6) speech, (7) gestures, (8) remote control, (9) keys and (10) touch. This may also
impact the distance between the interface and the display. Being aware of these
properties and their characteristics will allow designers clarifying the nature of
interaction they want to design for.
Display: These properties are mostly related to the technical details of large
programmable displays as described by Halskov and Ebsen (2013). They consist
of (1) type, (2) material, (3) shape, (4) size and (5) resolution but also of (6) the
type of content. Each of these characteristics can impact the multilayered spatial
frameworks described above (i.e. awareness, action, actor, physical).
Context: Understanding the socio-spatial settings as described in Sect. 2.1 are
core properties when locating a MAI. As explained in this paper, here is a huge
difference when designing in the context of a pedestrianised university court (i.e.
VEIV London) or a busy high street (i.e. SCSD Sao Paulo).
Although most of the characteristics aligned in our taxonomy describe the
properties of the constituent elements (i.e. display, interface, context) from a technical
point of view, they influence the multilayered spatial frameworks (i.e. awareness,
actor, action, physical) and consequently assist when designing MAI for
interactions in urban space. Through understanding the effect of the single properties
of each constituent element on the whole interactive system, designers may
eventually be able to be fully aware of their design decisions.

\subsubsection{Conclusion}
In this paper, we ask the question ‘how to go about designing for an interaction
with such a large programmable electronic display’ and introduced the
notion of MAI founded on existing spatial interaction frameworks established
in HCI research in recent years. We further described two design studies, which
we designed, deployed and observed in urban space. The first study was carried
out at a gathering in a university courtyard (i.e. VEIV London), whilst the second
was conducted in a complex urban setting in the city of Sao Paulo (i.e. SCSD
Sao Paulo). Further, we described the setting in which the interactive system was
arranged and presented our observations concerning mediated social interactions
in the surrounding space. Based on our findings, we discussed the relation of our
observations to four existing spatial frameworks and suggested a taxonomy, which
categorises the properties and characteristics related to our notion of MAI.
In this work, we laid the focus on social interactions mediated through our
MAI. For future development, we aim to explore other MAI, study their properties
between human behaviour, the user interface and the display in a given spatial layout
and compare them with our taxonomy.

\section{SCSD Linz}

\subsection{Ars Electronica Festival}


\section{Sentiment Cocoon}

\subsection{No.8@Arup Competition}

The competition, under which the Sentiment Cocoon evolved, called No.8@arup , is a programme that provides space to host installations and sculptures in Arup’s Central London office. The idea is a simple one, to provide an opportunity and encourage emerging architectural practices to showcase their creativity.
No.8@arup provides a fertile thinking ground to collaborate in a fast paced programme with support from Arup engineers. Arup is passionate about how the built environments will be shaped. Hence the creation of installations and sculptures plays a role in stimulating thinking around the transformation of our environment in the 21st Century.

The No.8@arup Installation is now in its third year. This year, entrants were asked to respond to the theme “Designing for People”, and consider how the No.8@arup installation should reflect the need for occupants to be placed at the heart of a design brief (see appendix). Arup aspires to create efficient and comfortable environments, where people are productive and also inspired, motivated, healthy and happy.
Previous installations have included the 'Balls'  (2014), which allowed visitors to interact with a moving sculpture and the ‘Splinter’  (2013), an 18m tall timber tower.

\subsection{Initial design idea}

We propose an interactive cocoon weaved out of a translucent fabric that turns the atrium into a stage for social encounter. The aim is to foster the notion of an atrium as the social centre of a building. Our focus is on the exploration of architectural form, translucent materials and responsive lighting to facilitate social interaction. We collect people’s sentiments and materialise them into light and fibre.
Based on our experience in human-computer interaction and interactive lighting we suggest a system that lets occupants interface with the light structure by using any RFID card. Simple sentiment interfaces attached to the entrance barriers and rails on each floor allow employees to express their current sentiments. These interactive dashboards were designed with knobs, dials and buttons. Each day, Arup people will be encouraged to share their sentiments via one of the dashboards that are installed on each of the six office floors. As people approach the dashboard they will be invited to choose which mood they are in to record their sentiment of the day. People will operate a dial and this will identify their sentiment, happy, sad or indifferent. Individual swipe cards such as the London Oyster card will enable participants to submit their sentiment for the day. As everyone’s RFID enabled swipe card is unique this will allow the system to identify behaviour albeit anonymously. A sophisticated algorithm will feed participants’ feelings through the dashboard and these will be digitally projected into a light field created by LEDs that forms the spine of the cocoon.
Our system architecture allows for the tracking and displaying of several behaviours. These parameters are encoded into lighting patterns that suggest the collective mood of the building’s occupants. The sentiment engine (database) collects human data in the office building through the sentiment recorder. The Sentiment Cocoon is to represent a collective visualisation of how everyone is feeling in the Arup head quarters in London on any given day. 
The lighting design of the Cocoon will create an enigmatic display. Natural daylight, pooling into the atrium from the skylights above will blend with the light emitted from the LEDs. This will allow for a rich interaction of varying forms of light, which will be diffused through the skin of the cocoon. The translucency of the material will create an effect whereby the suspended Sentiment Cocoon will generate a striking visual display of light that has been informed by the feelings of people working at Arup.
The sentiment will be encoded in different colours and movement. Colour gradients, the velocity of the colour and movement will be represented through patterns. Over time the patterns will become recognisable and therefore people working in the Arup building No.8 will experience the overriding sentiment of the day within the office.

The Sentiment Cocoon is yet another example of the increasing proliferation of media architectural interfaces mediating human behaviour with architecture. These social applications will ultimately lead to responsive, adaptable and clever buildings that serve human well-being.

\subsection{The realisation phase}

After we won the competition the hard work began. Within ten weeks we actually had to deliver a physical version of our initial design idea. Part of this challenge was the extremely short period of time that was left until the opening as well as the amount of research and development that was still needed before actually being able to build a physical form. 

The following timeline shows the structure of meetings Arup requested in order to deliver certain milestones at certain dates.

Timeline of realisation 10 weeks:
•	Friday 27th February: Kick-off meeting
•	Monday 9th March: Design meeting
•	Monday 13th April: Design review meeting
•	Friday 8th May – Sunday 17th May: Construction phase
Deliverables:
•	Design drawings of physical structure
•	Interaction design method
•	Lighting design 
•	Method statement
•	Budget estimates
•	Risk assessment (health and safety) 
•	Fire assessment

\subsection{The Design Process}

\subsubsection*{Form finding}

The idea for the Arup atrium was to design a continuous and organic cocoon like lightweight structure that would winch through the open space and connect all eight floors of the rectangular atrium.
For the initial form finding process we therefore developed a parametric programme that enabled us to explore different shapes and structures of the Cocoon. One structural objective was to determine the amount of splines on the outer surface needed to provide an optimized grid-sized surface for the transparent skin to stay tight. From physical prototyping we knew that the rhombus shaped grid’s size should not extend 50cm in its widest distance.
Based on the programming platform ‘openframeworks’  we were able to change several parameters through a graphical user interface (GUI), such as the amount of splines, the gradient of the splines or the direction in which the splines winch up. Even though parametric modelling has been conducted, extensive prototyping was needed to develop a cybernetic system that appeared to be interactive, lightweight and translucent.

\subsubsection*{Structural design}

Soon after the kick-off meeting and in collaboration with an Arup structural engineer we searched for possible structures that would represent the initial design concept. As our intent was to explore architectural form, translucent materials and responsive lighting to facilitate social interaction, we studied a series of construction principles that would represent the lightweight character of the proposed Cocoon:

•	Pre-fabricated metal spline structure
•	Pre-fabricated plywood spline structure
•	Metal tube loop structure
•	Plywood branch to core structure
•	Metal spoke to core structure

Metal spoke to core structure  
Based on the branch to core principle in the last section, engineers at Arup suggested to use threaded rods instead of timber, which require fewer diameter to deal with the same amount of loads. Eventually this structural option turned out to be the most feasible to start with. 

The first three structural principles follow the idea that the outer structure (i.e. the splines) are self-sufficient, meaning that all loads, such as the weight of the structure itself as well as the applied translucent fabric, will be carried through the surface structure. We were in favour of this idea, but considering planning, manufacturing and assembling efforts, eventually we decided to split the structure into an outer and inner configuration. Later it even turned out to be a good decision as during the prototyping phase we explored additional loads caused through the processing of the translucent outer skin.        

The scale of the Cocoon, initially proposed, was to have a 30m tall structure with the widest diameter of 5m within a 6m by 9m wide and 30m tall atrium. When considering all design constraints such as suspension of the installation, fabrication of the substructure and the existing staircase between ground floor and basement we came to the conclusion that the initial proposition needed to be down scaled slightly. The final size was 20m in height and 3.5m in diameter.

\subsubsection*{Translucent Skin}

Initially the translucent skin of the Cocoon was thought of a fibre structure applied onto the substructure through a spinning weaving machine. A small-scale prototype of this processing machine was already successfully in usage, however the enormous upscale of the fabrication process in connection with the available budget and the short developing time created insuperable challenges. As a consequence we had to rethink the making of the translucent skin from scratch. The new design intend here was to come up with a material that has very similar properties as the initially suggested fibres but at the same time the fabrication process needed to be significantly faster. After intense research into materials and production methods we finally came up with the – at first glance rather unattractive looking – cling film. Cling film, also called plastic or cling wrap, is an affordable and efficient material mostly used in logistics to wrap goods on pallets.

Consequently along our research journey several processing methods came up, as well as appropriate machines available on the market. Due to their weight and limitations in size, rather than freestanding machines, we decided to explore autonomous wrapping machines. After testing one of those machines, we eventually bought an affordable second hand pallet robot in Munich. In the next step we were able to hack the robot’s electronic in order to make customized processing programmes. Therefore we attached a micro-controller interface (i.e. raspberry pie) to remotely control the robot’s activities and to emergency stop it. Finally the pallet wrap robot was able to perform the wrapping of cling film up to two meters at any given diameter following a given outline. Furthermore the programmed weaving patterns allowed performing various patterns. 
As our initial design intents included the fabrication of the translucent skin on site, we were working on a processing method that would allow the weaving during working hours in the atrium space of the Arup London office. The aim was to get the fabrication out of hidden workshops into the heart of engineers and designers. Hence we planned to build a stage above ground level with the aim to bring the process of making as well as the digital fabrication through the robot on a plinth. From below onlookers would get a new perspective onto the production process. Implementing this intent required careful planning in close collaboration with the health and safety department at Arup to avoid distraction or even hazards for employees.

\subsection{Interaction Design}

The motivation for the interaction design was to develop, design and deploy a situated and tangible system that mediates collaborative interactions in public spaces whilst focusing on accessibility and affordance. In other words, the interface should be understandable and easy to use for people. The employed technology makes use of existing ‘Radio frequency identification’ (RFID) as known from smart card technology such as the London Oyster card. We built on the widely spread use of these unique ID tags for payless cash and travel purposes, as a large proportion of citizens in London carries an Oyster card in their pocket most of the time. Consequently the use of these cards is a recurring embodied interaction in the smart city. At the same time every interaction is uniquely identifiable and therefore traceable. Our aim was to allow people to use their ID tags beyond technical purposes and express their mood and opinion about specific issues in the technology mediated workplace. Hence the Sentiment Dashboard enables employees to express their current mood when in the Arup London office. 

The dashboard works as follows: On top of the dashboard the user is asked: ‘How are you feeling today?’ Below three categories were defined as follows: 1) Happy (green), 2) Motivated (blue), 3) Workload (red). By pushing one of the buttons on the device participants are able to choose one of the aforementioned categories. In the next step they need to turn the dial to express if they are happy, indifferent or sad and eventually swiping their RFID token (i.e. Oyster card) across the bottom of the interface where it says: ‘Register your Sentiment: Swipe your Card’ to submit their mood to the Sentiment Cocoon. In summary, the sentiment dashboard affords three folded interactions: 1) pushing: one out of three categories can be chosen; 2) dialling: by turning a rotary switch the user can gradually adjust their mood; 3) swiping: finally the electronic ID card needs to be swiped over the allocated field.
The interaction modus was set to allow participants to only express their mood once an hour. This measure has been introduced to avoid certain users swiping over and over again to manipulate the results. 
In total six dashboards were distributed across six floors of the office building. All dashboards were installed at the balustrade facing towards the atrium. 
In the heart of the interactive dashboard there is a microcontroller (i.e. Arduino Yun), which can be directly connected via an Ethernet cable to the in-house network. On top of the Arduino Yun we placed a customized sentiment shield, which we have designed and produced for the specific needs of the sentiment data collection. Each shield is equipped with three RFID reader, three slots for potentiometer input and three slots for push button input.

\subsection{Lighting Design}

For the lighting design the idea was twofold, to visualise the data captured by the six sentiment dashboards and to also create a visceral response to the interaction with the dashboard. The intent has been to use the lighting within the Cocoon as a visual indicator that represents and physically situates the recorded ‘sentiments’ within the Cocoon and thus in the ‘real’ space of the Arup atrium. As well, the lighting was intended to mirror the actions, and the physical presence of people interacting with the sentiment dashboard that is connected to the Cocoon sculpture. When a user swipes their RFID card and ‘sends’ (records) their sentiment a white flash of light appears directly in front within the Cocoon. The white flash of light is intended to mirror the users presence and to give them the feeling that they have transformed their sentiment and presence into a pulse of light. The intent of the visceral lighting response was to encourage and to provide a moment of instant gratification, which in turn would encourage further interaction. It was also important the lighting visualisation not be too detailed, that users could immediately intuit the meaning of the lighting.

The LEDs are four continuous lines totalling 4800 pixels that generate complex patterns and gradients of colour. Running the entire height of the Sentiment Cocoon, 20 metres, the LEDs will create an enigmatic display. Natural daylight, pooling into the atrium from the skylights above will blend with the light emitted from the LEDs. This will allow for a rich interaction of varying forms of light, which will be diffused through the skin of the cocoon. The translucency of the material will create an effect whereby the suspended Sentiment Cocoon will generate a striking visual display of light that has been informed by the feelings of people working at Arup. The sentiment will be encoded in different colours and movement. Colour gradients, the velocity of the colour and movement will be represented through patterns. Over time the patterns will become recognisable and therefore people working in No.8 will experience the overriding sentiment of the day within the office.
The lighting visualisation was implemented using an LED lighting system, which consisted of a power and control system along with four 20m runs of LED ribbon, which can be controlled down to the level of each individual pixel. A bespoke app, known as the “light server” was created to control the lighting within the Cocoon, running within the app is an algorithm designed to record, catalogue and transform the individual interactions (recorded sentiments) into animated light pulses. Interactions are recorded via Arduino Yun micro controllers situated within each dashboard that send data recorded via interactions to the app. Another aspect of the implementation was devising how to blend the emitted light with the physical materials used in the Cocoon, specifically the plastic wrap skin and light diffusion materials, the goal was to create a blending of light and materials that were indistinguishable from each other.

\subsubsection{The Construction Phase}

\subsubsection*{The Final Cocoon Structure}

The Sentiment Cocoon is an in total 20m tall and in between 1.4m and 3.5m wide lightweight structure suspended from the atrium ceiling. 
Alongside an inner core with a diameter of 0.15m there are 21 rings that consist of ten spokes, each layered on top of each other in a distance of 1m. The inner core is made off 20 segments each 1m high. One segment consists of two plywood disks, which host the horizontal spokes and two vertical threaded rods. In between the two plywood disks there is a 0.96m long white PVC tube (thickness 1.5mm). The two threaded rods (10mm) compress both disks against the PVC tube. All segments are interconnected and therefore require onsite construction.
The horizontal rings consist of ten threaded rods (thickness 10mm), which are screwed into the plywood disks. During the prototyping phase the applied forces onto the larger rings (i.e. radius > 1m) required to support the threaded rods with additional metal tubes.
All the way from top to bottom the 210 spokes are tied together with 2mm metal ropes, in a rhombic pattern. At the position where the metal ropes meet each spoke they are clamped together with a round-head screw. Snap hooks on both ends of the ropes connect them back to the inner core.
The whole construction was designed to be suspended from the atrium’s ceiling. Hence special construction efforts were needed. During the initial design stage it was planned to build a truss structure on top of the 5th floor. After first cost estimates building facilities and Arup engineers decided to make use of the building maintenance unit (BMU) on the 6th floor. Usually the BMU is pulled out to clean the glazed parts of the atrium. After structural considerations it was possible to attach the cable winch to the BMU. Being able to use the BMU enabled us to navigate the Cocoon in all directions (x-, y-, z- axis).

Another challenge that needed careful considerations during the design construction process from the very beginning on was the restriction of access for production materials. The only available entrances into the building were limited to the dimensions of 2.2m in height and 1.4m in width. As a matter of fact it was not possible to prefabricate the Cocoon and deliver it in one piece. Even the delivery of single segments was not feasible. Therefore all parts had to be delivered individually and assembled on site. Assembling on site compared to prefabrication on site requires a much simpler method as the processing conditions differ enormously (e.g. skilled worker versus motivated volunteers). Eventually the whole substructure (i.e. the inner core, the spokes and the metal ropes) was assembled only with hand tools.

\subsubsection*{Setting up the Cocoon}

The on site construction started Friday May 8th in the evening with the setting up of the scaffolding platform founded on the basement floor and finished with the upper edge on 1.4m above ground level. To ensure safe working the platform needed to be secured with a balustrade and people had to wear full personal protective equipment (PPE) whilst on platform. In addition everyone involved needed a induction to Health and Safety as well as fire safety standards held by Arups H and S officer.
Before the scaffolding platform could be completed, the wrapping robot had to be lifted through a whole in the wooden blanks with the help of the BMU. Thus the rigging company had to attach the winch to the movable truss of the BMU. In the next step the robot could be lifted onto the platform. After this the platform floor was closed and additionally the raw scaffolding floor needed to be covered by additional MDF boards to provide the pallet wrapping robot with a smooth surface. 
Another task during the pre-setup included the construction of a wooden ring with a diameter of 3.7m in the centre of the platform. The outline of this ring gave the wrapping robot the direction to follow. The centre of the ring carried a metal mounting fixture to hold the different segments of the Cocoon whilst assembled.
The pre-setup was accomplished by Saturday and the construction of the Cocoon itself started. First of all the winch was lowered onto platform level in order to equip the hook with the first segment of the inner core. At the same time all power supplies and data wires for the LED stripes had to be mounted on a separate winch and connected with the top of the first segment. The aim was to lift both the power supply pack and the Cocoon at the same time. Therefore all electrical installations had to fully function before lifting, as maintenance work afterwards would have been difficult. The next steps included the mounting of the plywood disks with the threaded rods and the PVC tube in between the disks. The four LED strips were fixed to the white PVC tubes of the inner core. 
Two segments consisted of three rings and two PVC tubes and formed one unit that the pallet-wrapping robot was then able to wrap. All rings had to be assembled on the platform.
Only on Monday we were able to wrap the first segment and hoist it. From then on we tried to speed up the process in order to achieve our goal to finish all ten segments by the end of the week on Friday evening.
On Saturday the dismantling of the temporary support structure began. First the robot had to be lifted off the platform. As the winch was in use for the Cocoon riggers had to bring in a second winch and set up a truss frame to lower the robot from the platform back on to basement level. After this the scaffolding company dismantled the platform. 
All works finished in time before Monday morning the office opened as usual. During the following two weeks we had to prepare the dashboards, set up the network and finish the interactive part of the installation.

\subsubsection*{Dismantling the Cocoon}

The disassembling of the Cocoon started on Friday 28th September and finished on Sunday 30th September. This time there was no scaffolding needed on the ground floor level. Due to the winch fixed to the BMU the riggers were able to navigate the Cocoon into one corner of the atrium. This way the structure did not conflict with the existing staircase when lowering the Cocoon. However, the floor on the basement needed to be covered and barriers were set up to prevent unauthorised entering of the construction site. 
The Cocoon was carefully lowered onto working height before volunteers cut off the cling film and the metal ropes. In a next step the diffusers were dismantled and the subjacent LED strips were pulled off. Only then we screwed off the threaded rods and the plywood connectors. A second team of volunteers unscrewed the nuts and separated the threaded rods from the metal tubes. Basically three types of materials needed to be recycled: plastic (i.e. cling film, diffuser film and PVC tubes), wood (i.e. plywood connector) and metal (i.e. threaded rods, metal ropes and tubes). As a consequence all materials could be separated and recycled unmixed. The dismantling of each of the twenty segments was finished by Saturday afternoon. On Sunday the riggers took the winch off the BMU and got the cleaning unit back onto the BMU.

\subsection{Findings}

The installation was running 24/7 throughout 13 weeks in total starting Monday 25th of May 2015 and finishing on Friday 28th September 2015. During this time the database logged about 1880 single interactions, which were recorded through six sentiment dashboards. Many employees at Arup on their way to their desk, during lunch time or in the late afternoon on their way back home either engaged directly with their Oyster cards (direct interaction) or simply enjoyed the colourful and dynamic visualisation on the Cocoon (ambient interaction). According to the collected data most interactions took place after the official opening of the Cocoon (June 2nd 2015), after the Show and Tell event (June 16th 2015), during the opening of another exhibition in the Arup reception space (June 22nd 2015) and whilst the Arup summer party was in full swing (July 14th 2015).

We conducted a preliminary data analysis of the 1880 valid ID card interactions captured by the dashboards and logged on our database. 

The aims were two folded:
•	Understanding how participants use the dashboard. 
•	Identifying sentiment patterns of group and individual behaviour.

For each unique card ID that has been used, we looked at the logged data sets and extracted the specifics of the submitted sentiments (which floor, which category, and which preference to this category). 

From our observations and in accordance to the collected sentiment data three different major participation patterns were observed when individuals or groups approached the sentiment dashboards: 

•	Serious behaviour:
o	This was the least frequently identified participation pattern. The participant (card ID) has submitted exactly one sentiment for each of the explored categories. And each submission recorded a different preference value (i.e. in between 0-1024). This pattern would reflect how we expected the interaction mechanism to work - i.e. a person would explore the categories by pushing one of the three buttons and would submit one sentiment for a specific preference. 
•	Clumsy behaviour: 
o	This was the most frequently extracted participation pattern. The participant (card ID) has submitted the same sentiment (preference) for each of the three categories. Or the same value of the sentiment was submitted as by the previous user. The occurrence of this pattern can be explained with our frequent observation of participants holding their card over the RFID reader only without having pushed any of the category buttons. This behaviour might be due to a usability flaw of our installation - the participating person did not realise the effect of her participation in the visualisation, hence tried several times. 
•	Playful behaviour: 
o	The participant (card ID) has submitted several different preferences for the same category within the considered period of time. This might indicate that s/he did not really want to express an opinion, but rather explored how the installation and the visualisation work. 

After eliminating the repetitive submissions, we extracted the distribution of submissions across the three categories. The category “workload” was the most popular (35 percent) of all submissions, followed by “motivation” (33 percent) and “happiness” (32 percent). While we cannot account for representative polling results, the findings indicate the installation fulfilled its intentions as a public feedback platform, where people engage meaningfully with their sentiments. Besides the data captured through the TUI device we observed interactions around the Cocoon in the close interaction space as well as in the wider ambient space. Although these observations were not rigorously conducted, we did notice a few recurring behaviours. In particular, we frequently saw people taking pictures of the installation with their mobile phones or taking pictures of each other in front of the Cocoon. In addition we observed Arup employees introducing the Cocoon to their clients when they are visiting the building. These informal observations would suggest that people liked the Cocoon and the visualisation it used.

\subsection{Discussion}

In the previous sections we have introduced our research interest in interactive media architecture, gave insight into the design and building process of our case study the Sentiment Cocoon and provided a first glance on the freshly gathered data.
In this section the discussion about commercialising academic research requires the consideration of diverse aspects one usually is not immediately concerned in academic research.

\subsubsection*{Initial design}

The nature of any initial design proposal is that it will need much iteration and some alterations when it comes to the construction design phase and will need to be adjusted to the constraints and limitations in the real-world context. For instance in the case of the Cocoon we had to downscale the height and diameter of the Cocoon’s outer shape. Otherwise fabrication on site with the help of the pallet-wrapping robot would not have been possible. On another note, our ambitions of creating a 3D printing machine that weaves the Cocoon’s skin through fibres in one piece had to be re-engineered considering the limited budget and the extremely short tie for research and development. The final outcome was a pallet-wrapping robot that would fabricate the Cocoon’s outer skin with cling-film in segments. However, we do not think that these alterations downgraded our initial design; instead, we demonstrated that we were able to adapt our methods for the good of our design vision as well for the real world implementation. 

\subsubsection*{Construction}

To come up with the final design construction for the Cocoon much iteration was explored on paper however only physical prototyping eventually led to the final design. Looking at the construction of the Cocoon and its scale, we underestimated the high standards towards health and safety when it came to the actual implementation on site. At  the same time the budget that would be required for external contractors to set up the construction site (i.e. scaffolding and rigging) blow our initial estimates. Considering that most of the work contractors were assigned for was merely allowed to be executed by qualified and certified staff there was hardly any possibilities to optimise these estimates. It is to say that due to Arup’s enthusiastic attitude towards our project the budget was raised by almost one third. As a result we were able to cover the additional costs by our contractors.
Furthermore we underestimated cost and effort to dismantle the structure again. The fact that everything was assembled on site made it easier to dismantle all parts. Arup requires high standards for sustainability and therefore recycling needed to be done properly.   
In addition the amount of paper work that was compulsory in order to get permission for building the Cocoon inside an office space (i.e. fire assessment, risk assessment) could only be dealt with due to the help supported by Arup. In summary it is to say that we clearly underestimated the ‘hidden’ costs in our initial cost estimate.

\subsubsection*{Interaction Design}

From the technological perspective the Sentiment Cocoon set up consisting of six networked sentiment dashboards, data collection and the connected light visualisation worked perfectly. We have not had a single outage so far, neither from the hardware (i.e. dashboards or LED installation) nor the software. The reason for this is that we put enormous efforts into the development of the networked sentiment dashboards and the visual representation of the collected data by employing two experienced developers. At the same time we were able to report on our previous experiences with the deployment of feedback dashboards for the developers to start from. This is a great improvement in the development and a further step towards commercialising the idea of building a platform that connects an ecology of sentiment interfaces. 
However seen from longitudinal aspects, we have noted that interactions have dropped down considerably over the last couple of weeks. The reason for this could be that many users now seem to understand how the Cocoon works and may have become bored with it. This would mean that the only attraction would have been the novelty effect. On the other hand one may argue that interactions will need to level off. The data sets suggest that this process started after week five of the deployment. In the following weeks peaks could be identified when certain events happened, such as the Arup summer party on Tuesday July 14th. 
In particular the transition from the novelty effect to the levelling off phase is a highly relevant observation that clearly needs further exploration.

\subsubsection*{Lighting Design}

From the point of view of fulfilling the original design intent towards the lighting design the outcome was successful in terms of technological implementation as well as from the user experience perspective. The technological implementation was carefully planned from the very beginning and managed sensibly during the construction phase. Only one time a serious issue appeared when the LED strips started to flicker. We discovered the lighting within the cocoon would flicker rapidly whenever there would be a lot of brightness and movement in the lighting patterns. From previous experience of running similar installations with short data/power cables we knew immediately this was due to the length of the cable runs from the control and power supply to the LED lights but were not sure of the exact cause. We found the answer quickly online in the knowledge base associated with the control card we are using. It turned out that the pluggable resistor network on the control card was the cause. The resistor packs on the control card were changed to eliminate the flicker issue as proposed by the online solution
 
In tackling the challenge of making the lighting work we learnt much more technically about the circuitry and operation of LED control systems and the DMX protocol. In making the system work properly we also learnt about TCP/IP protocols and UDP packet structure and its relationship to high frame rate capability in the animating of LED lights. The online knowledge base, in this case what is most interesting is that the control card we are using runs the DMX/Artnet protocols standard to the lighting industry and normally costs upward of £3000. However, there are hobbyists using the same protocols and have created their own hardware. What is available online is a card worth £600 and by consequence a knowledge base around it that proved to be very valuable on this project.

The design intent of the light visualisation has been accomplished, but more complex behaviours could not be implemented due to time and budget reasons. Over time the users seem to have come to know the Cocoon visualisation and are in a position to have more curiosity awakened by the behaviour of the Cocoon evolving over time. This again might be a reason for the drop of interactions after week five. We also learnt that to achieve both good interaction and visualisation one needs to be weary of ‘feature burden’ where the interactions and visualisations are too complex for users to understand or relate to. Observations in relations to this were made regularly. Many users did not understand the operation procedure when approaching the interaction device. For instance we experienced people who swiped their RFID card before actually turning the dial or pushing a category button. At the same time many people understood that their actions on the dashboard will trigger a change in the lighting behaviour of the Cocoon, but could not explain what it mean. 

In summary it is to say that even the fact that our strong design intent always was to focus on simplicity, when it comes to interaction modes and visual representation of those interactions, we still need to improve the clarity the dashboard and lighting design.

\subsubsection*{Research and Development - Prototyping}

Due to time and budget constraints research and development (R and D) presented many difficult challenges, there was seldom enough time for prototyping the interaction of the materials and the lighting, this could have been much better. As well, there was little time and money to facilitate the proper testing of the power system for the lighting. We have clearly underestimated the amount of resources needed for the prototyping stage. 
In essence, extensive prototyping in advance can save both time and money, arguably it may even save more money than it costs to prototype. 

\subsubsection*{Public relations and academic dissemination} 

Crucial to any commercial and research project no matter how successful it is, is to talk about it and spread the word. This needs to be planned well in advance and with the same care as other design work packages within in a project.   
From the beginning on Arup provided a film making team that accompanied us from day one of the building process on site and left when the Cocoon’s scaffolding has been removed. In addition we asked a filmmaker known to us to take additional footage during the setting up and to take shots when people interact with the Cocoon. Throughout all stages of this project we took pictures with mobile phones and cameras.
However we had to learn that we did not budget enough for the documentation and dissemination process – also, a clear strategy was missing – we were relying on Arup as they promised a PR agency. Their outreach was limited and reached only one print magazine and a few less popular design blogs. Only now we are trying to research relevant magazines, blogs and other platforms through an additional PR person with an updated story and better audio-visual coverage.

\subsubsection*{Collaboration, team and social dynamics}

Generally speaking the collaboration between a small creative practice and a large resourceful and knowledgeable corporate global company should ideally be of mutual benefit towards prototyping our future city. One may argue that young practices bring in fresh ideas into creative processes, which ideally flourish through the secondment by experienced and established companies. However it comes with some challenges both sides need to be aware of in order not to discourage the others. We were constantly appreciating the expertise and wide experience of Arup engineers and the open mindedness of Arup with regards to exploring new directions in design and architecture. Alongside this we had to accept that the corporate space of a global company his highly controlled in the way that communication leaving the bespoke space needs to be signed off by company officials. This can be cumbersome from time to time as young practitioners are constantly in the need of sharing their activities in form of snapshot images or video footage with their digital networks in order to increase their professional exposure.   

From the beginning of this project the core team consisted of three members; an architect, a lighting designer and a product designer. After we won the competition and the scope of work was clear there was the biggest moment of uncertainty in this project. We had promised a huge design vision for which we did not know how to deliver at this point. This entrepreneurial risk seemed to be too high for one of the core members. As a consequence and after intense attempts to keep him on board the product designer, who was responsible for the digital fabrication, left the team. This caused a vacuum that needed to be closed rapidly. After intensive search we were not able to find a full replacement and decided to split the tasks amongst the remaining two project partners in order to distribute the risks of another failure. Having learnt from this we found it extremely hard to bring a team together that would stick together no matter what challenges may suddenly appear. We found it easy to get people enthusiastic about our project and of course to work with Arup, but finding people with the right skills and commitment was almost impossible within the given time and budget. We learnt that we needed to come up with intangible incentives such as individual creative freedom, access to a professional network and social amenities to convince enthusiasts to come on board. At the same time besides the efforts to find the right kind of collaborators, the core team constantly had to keep the initial design intent in mind to avoid the leaching out through too many concessions towards individual creative freedom. 

\subsubsection*{CManaging expectations – the client relationship}

As a matter of fact the different parties involved in this project wanted to achieve the same goals (i.e. completion of the Sentiment Cocoon) their motivations however were different. For Arup it seemed to be the priority to showcase the Cocoon as a structure as well as a vision for better workplaces to their clients in order to demonstrate Arup’s ability to innovation and to support new designers through this competition.
For us as the designers and researchers it was preliminary important to materialise and implement our vision and the containing research knowledge. And of course to increase our professional network, by showing our skills. 
At the same time, managing expectations was a tricky venture throughout the project. For instance when initially proposing a 3D printer during the competition people actually expected what they considered a 3D printer. When we came up with a pallet-wrapping robot, they were rather disappointed of this creature that looked more like a vacuum cleaner. However, considering the fact that the robot quickly got a name (i.e. Einstein) seemed to be a sign of some kind of affection. In addition, calling the new surface a cling film structure made the whole process of wrapping Cocoon sounded a bit trivial. 
Another issue rose when we were calling the pallet-wrapping machine a robot. The health and safety department was particularly concerned about workplace safety. An autonomous machine in an office space would have been an uncontrollably risk. Only when we demonstrated that this machine is not acting like an industrial robot arm being able to cause severe damage we were able to convince the person in charge to sign off this fabrication process.
In a nutshell, the challenge for a successful client relationship is a careful management of expectations.

\subsubsection{Conclusion}

With the Sentiment Cocoon we have achieved both an interesting aesthetically pleasing installation that can encourage interaction. Through its implementation in an atrium surrounded by workplaces, the sentiment data collected by the dashboards created a value. It gained relevance towards enhancing the wellbeing of employees. In other words we successfully embedded academic knowledge into a commercial use case. In addition we have created a discussion about the mapping of emotions relative to a specific location (i.e. workplace) as well as to the happenings in that location. We strongly feel we have moved toward an interesting phase in the use of aesthetic architectural works to trigger and enliven contemplation of social conditions and issues directly related to specific locations. Eventually we can assume that there is a demand for situated sentiment analysis in a workplace.
From an architectural point of view we have described the design process from the very beginning of the competition up to the opening of the interactive structure. The aim here was to share the enormous efforts - mostly invisible – we undertook to successfully complete such a venture. We have discussed the challenges and lessons learnt. Finally this may help other academic researchers from the emerging fields of interactive media architecture to explore the opportunities of commercialising their work.
Speaking for ourselves, due to the exciting outcome and huge success of the Sentiment Cocoon, we are currently exploring options to further push our academic research towards applied practice.







