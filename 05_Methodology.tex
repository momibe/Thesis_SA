\chapter{Methodology}
\label{chapterlabel4}

\section*{Chapter summary}

In this chapter I will introduce the empirical methodology for this exploratory research. A mix of existing cross-disciplinary approaches have been applied, for instance, "research through design", iterative urban prototyping,   \newpage



\section{Research through design approach}

This PhD research is located within the fields of architectural research and human-computer interaction research in urban space (urban HCI). Both fields are concerned with human behaviour, urban space and increasingly with digital media technologies (i.e. media architecture). This PhD research explores the design, deployment and evaluation of ‘media architectural interfaces’ (MAIs) in urban space. ‘Media architectural interfaces’ (MAIs) consist of a mediator (i.e. situated ‘tangible user interfaces’) and a carrier (i.e. urban screen or media facade). More specifically, this PhD research will explore, how citizens may use the mediator to engage with the carrier. And further how the deployment of ‘media architectural interfaces’ (MAIs) may influence the socio-spatial configuration of pedestrian flows, encounter and social interactions.
In the following this PhD research suggests a methodology, which brings together empirical architectural research and interaction design research (O’Neill et al., 2006; Fatah et al, 2013). It is driven by a research through design approach (i.e. iterative prototyping) using qualitative and quantitative methods.

The methodology will focus on four goals in order to address the research: 
\begin{enumerate}
\setcounter{enumi}{0}
\item Design and deployment of the ‘media architectural interface’ MAI (section link);
\item Analysing full-body interactions of participants interacting with the ‘media architectural interface’ (section link); 
\item Identifying participants behaviour when bridging the focal awareness space at the TUI and the distant awareness space at the media façade (section link);
\item Observing and analysing of pedestrian flows, encounters and social interactions before and during the deployment of the ‘media architectural interface’ (section link);
\end{enumerate}

To clearly describe the methodology a chronological approach subdivides it into three stages. The pre-deployment stage marks all research activities that are required prior to the actual setting up of the ‘media architectural interface’ (MAI). It mainly involves the search for an appropriate location that accepts the Mediator (i.e. tangible user interface) and a Carrier (i.e. media facade) to connect to. To inform the appropriation of the location and to have comparative data sets for the evaluation afterwards, spatial observations (see next section) have to be carried out in the field. 

The during-deployment stage will mostly take place on location. The MAI is fully working and the data collection starts. As Mediator and Carrier are connected through a database, log files will store information about the interactions between user and the Mediator. Observations about the types of encounters the Attractor enables will be captured in the direct space around the Mediator and in the wider surrounding that is affected by the Carrier. In addition, audio-visual data such as full-body interactions will be captured through video recording and picture taking. 
In the after-deployment stage all collected data will be analysed. The data gained through spatial observation will be mapped and compared to the observation data from the pre-development stage.


\section{Iterative urban prototyping}

The prototype will be based on a technology that has been developed during a series of iterative design processes before. The shared and situated tangible interface has been originally developed as part of the authors MSc studies and since then been iteratively deployed, tested-out during various events and improved (i.e. Behrens, 2011; Behrens, 2013). The technology facilitates on the ubiquitous travel cards, such as the London Oyster Card, which is based on RFID technology. The goal is to make use of existing smart technology beyond travel purposes and allow citizens to express their mood and opinion instantly in the technology mediated urban realm.

\section{The Interface}

As outlined in the introduction, the MAI consists out of two elements: 
Mediator is a situated ‘tangible user interface’ that generates user specific data 
Carrier is the object that displays the data generated by the Mediator. 
The carrier might be a building projection, an urban screen, or a media façade and might change depending on the location of the case study. The Mediator is built on the same technology (fig. 5) and will not change. Only the appearance (i.e. case or description) might change according to the content of the deployment (fig. 22).
For this PhD research the prototype of the Mediator will be based on a technology that has already been developed during a series of iterative design processes. The system architecture behind the ‘situated tangible interface’ has been originally invented as part of the authors MSc studies and since then been iteratively deployed, tested-out during various events and improved (i.e. Behrens, 2011; Behrens, 2013). The system is based on ubiquitous travel cards, such as the London Oyster Card, which are based on radio-frequency identification (RFID) technology. The idea is to build on the fact that almost everyone is carrying around such an identifiable tag and make use of the existing smart technology beyond travel or access purposes and allow citizens to express their mood and opinion instantly using this technology in the urban realm. Using an RFID tag will allow to gather data (fig. 6) such as preference of interactions (1), identification of interaction (2), date (3), and time (4) of interactions.The TUI is connected to a database, which logs all data. The visualization for the carrier will pick up the data from the database. Interactions with the feedback device will be plotted quantitatively onto timeline diagrams  (fig. 7), which show the ID number of the participants transport card, the time when the interaction took place and the preference of the interaction (i.e. like or dislike).
The outcomes of the surveys will be visualized and qualitative observations and interviews will be summarized in written form.


\section{Data collection, observations and analysis}

Before actually deploying a prototype interface, spatial observations need to quantify and qualify people’s behaviour and movement in relation to the specific layout of their built environment. The objectives are firstly to identify suitable locations for the deployment of the prototype and secondly to capture the socio-spatial configurations before the deployment in order to be able to compare them with the configurations before and during the design intervention. Spatial observations will be conducted as developed by Space Syntax and summarized by Kinda Al Sayed et al (2013). Therefore researchers will need to be in the field during all times of a weekday and a weekend day.