\chapter{Background}
\label{chapterlabel2}

\section*{Chapter summary}

In this chapter, I provide an overview of relevant projects and research
in the fields of media architecture and human–computer interaction and focus on
frameworks that describe social interactions mediated through interactive systems,
with a particular focus on urban screens and media facades..\newpage


\section{From anthropometric scale to sentiment analysis/architecture}

According to Pheasant and Haslegrave (2006) \cite{Pheasant2006} ''Anthropometric is a branch of the human sciences that deals with the body measurements, particularly with measurements of body size, shape, strength, mobility and flexibility and working capacity. Humans are variable (in dimensions, proportions and shape, as in all other characteristics), and user-centred design requires an understanding of this variability."

Anthropometric data according to Jeremy Stranks Human Factors and Behavioural Safety \cite {Stranks2007}

\begin{itemize} 
  \item sitting or standing height
  \item arm length
  \item arm reach in a forward, sideways, upward and downwards direction
  \item hand and finger size
  \item knee height
  \item strength of individual muscle groups
  \item range of movement possible at various limb joints
  \item grip strength
\end{itemize}

Behavioural data / sentiment / sensual data 
\begin{itemize} 
  \item 
  \item 
  \item 
\end{itemize}


''Designing for people" in the 20th century was characterised by Cartesian methods to map the physical needs of humans. Architects and designers such as LeCorbusier (2004), Dreyfuss (1955) or Neufert (2002) surveyed human ergonomics to suggest design principles for an anthropometric scale based on Vitruv’s early considerations. Today it is all about the understanding of human behaviour. The advent of the digital turned our built environment into an ecology of sensors measuring our behaviour. Human-computer interfaces such as mobile devices, urban screens or media facades augment our physical presence in digital space. But how do we actually design our physical space for people in the digital age?
We propose an interactive cocoon weaved out of a translucent fabric that turns the atrium into a stage for social encounter. The aim is to foster the notion of an atrium as the social centre of a building. Our focus is on the exploration of architectural form, translucent materials and responsive lighting to facilitate social interaction. We collect people’s sentiments and materialise them into light and fibre.
Based on our experience in human-computer interaction and interactive lighting we suggest a system that lets occupants interface with the light structure by using any RFID card. Simple sentiment interfaces attached to the entrance barriers and rails on each floor allow employees to express their current sentiments. These interactive dashboards were designed with knobs, dials and buttons. Each day, Arup people will be encouraged to share their sentiments via one of the dashboards that are installed on each of the six office floors. As people approach the dashboard they will be invited to choose which mood they are in to record their sentiment of the day. People will operate a dial and this will identify their sentiment, happy, sad or indifferent. Individual swipe cards such as the London Oyster card will enable participants to submit their sentiment for the day. As everyone’s RFID enabled swipe card is unique this will allow the system to identify behaviour albeit anonymously. A sophisticated algorithm will feed participants’ feelings through the dashboard and these will be digitally projected into a light field created by LEDs that forms the spine of the cocoon.
Our system architecture allows for the tracking and displaying of several behaviours. These parameters are encoded into lighting patterns that suggest the collective mood of the building’s occupants. The sentiment engine (database) collects human data in the office building through the sentiment recorder. The Sentiment Cocoon is to represent a collective visualisation of how everyone is feeling in the Arup head quarters in London on any given day. 
The lighting design of the Cocoon will create an enigmatic display. Natural daylight, pooling into the atrium from the skylights above will blend with the light emitted from the LEDs. This will allow for a rich interaction of varying forms of light, which will be diffused through the skin of the cocoon. The translucency of the material will create an effect whereby the suspended Sentiment Cocoon will generate a striking visual display of light that has been informed by the feelings of people working at Arup.
The sentiment will be encoded in different colours and movement. Colour gradients, the velocity of the colour and movement will be represented through patterns. Over time the patterns will become recognisable and therefore people working in the Arup building No.8 will experience the overriding sentiment of the day within the office.

The Sentiment Cocoon is yet another example of the increasing proliferation of media architectural interfaces mediating human behaviour with architecture. These social applications will ultimately lead to responsive, adaptable and clever buildings that serve human wellbeing. 

\section{From public displays to urban screens and media facades}

\blindtext

\subsection{DIY displays}

\blindtext

\subsection{Public displays}

\blindtext

\subsection{Urban screens}

\blindtext

\subsection{Media facades}

\blindtext

\subsection{Interactive media architecture}

\blindtext

\section{Urban IxD: Technology mediated interactions in urban space}

\blindtext

\subsection{From urban computing to urban IxD}

\blindtext

\subsection{Tangible user interfaces (TUIs)}

\blindtext

\subsection{RFID interfaces}

\blindtext

\section {Interfacing with Media Architecture}

Due to technological advancement, large public displays became ever more incorporated
into the built environment and because of price decline its application for
social and artistic purposes became popular in urban space. This led to novel technology-
mediated social interactions, such as people engaging with media facades
through tangible devices.
Tangible user interfaces (TUIs) ''give physical form to digital information,
employing physical artifacts both as representations and controls for computational
media" \cite{Ullmer2000}(p. 916). Tangible Interactions evolved from
research in TUI and rely on embodied interaction, tangible manipulation, physical
representation of data and embeddedness in real space and give computational
resources and data material form \cite{Hornecker2006}. The intention is to
embed computing into everyday life and support intuitive use.
One of the first social media art projects that provided an interface allowing
passers-by to create and share content on a media facade was the BlinkenLights \footnote{xxx}
project in 2001. Participants on a street in Berlin were able to share messages
typed into a mobile phone with the public through posting them on a low-resolution
media facade (each pixel was an illuminated window in an office building).
Over the past few years there have been numerous media art and research projects that have developed user interfaces and applications to transmit actions, in situ and in real-time, on to a media facade that is connected to an interface. For example, SMSlingshot \cite{Fischer2012}, Sonic Skate Plaza \cite{Serret2013},6 Binoculars \cite{Guljajeva2013},\footnote{xxx} or SCSD Sao Paulo \cite{Behrens2013}.\footnote{xxx}
More recently, multiuser interactions with media facades through mobile
devices revealed challenges when deploying interactive artifacts in urban space
that enable passers-by to engage with media facades \cite{Boring2011}. Wiethoff
and Gehring \cite{Wiethoff2012} introduced a design toolkit to prototype when designing interactions with media facades before the actual deployment. Since then, novel interfaces
have been designed and deployed in the urban environment that let people
interact with media facades \cite{Hoggenmueller2014}. We aim to contribute
to this existing research by exploring the relation between TUIs and large
programmable displays from an HCI as well as architectural point of view.

\section {Media Architecture and the Role of Context}

McQuire \cite{McQuire2006} argues that TV screens have been transformed from small-scale
interior devices to large architectural surfaces that no longer broadcast to private
inside spaces but to public outside spaces. Architectural surfaces turned into public
media interfaces, transmitting mostly content curated by corporate organisations,
rather than interactions generated by the public in situ. The monochromic
''Spectacolour Board" at the New York Times building, set up in 1976, is considered
to be the first large electronic display in urban space, which broadcasts
dynamic content \cite {McQuire2009}. At the time, this was a technological
advancement of the traditional billboard, serving to broadcast commercial content
\cite{Huhtamo2009}. From then on, the prevalence of urban screens was unstoppable
in particular technological developments and price decline accelerated this trend
in recent years. Whole building facades subsequently became digital walls such as
the facades surrounding the New York Times Square that display dynamic content.
Urban screens are either stand alone (e.g. large BBC screens initially initiated
by the BBC, in Liverpool, UK), attached onto existing building facades (e.g.
Piccadilly Circus, London), or the digital media technology is already weaved into
the building’s surface (e.g. the Galleria department store in Seoul, Korea). Visually
animated surfaces, such as dynamic light facades, are equipped with numerous
light-emitting diodes (LED). They turn into large programmable pixel matrices
displaying animated visual patterns. From a technical perspective, there are other
types of artificial light-based media facades as well, such as projections onto
facades, back projections through glazed facades, or three-dimensional media facades (i.e. voxel facades) \cite{Haeusler2009}. Some see digital media facades as
simple ornaments that create an ambient atmosphere \cite{Caspary2009}. Others consider
the potential of digital media facades for communicating content, for
instance, advertising (e.g. Times Square, New York or Piccadilly Circus, London),
news content (e.g. the network of Big Screens in the UK, which was initially run
by the BBC),\footnote{xxxBBCBigscreen} media art (e.g. Lozano Hemmer’s work),\footnote{xxx} social visualization (e.g.
BlinkenLights)\footnote{xxx} or for community purposes on a neighbourhood level
(e.g. Screens in the Wild).\footnote{xxx}
Extensive research has been carried out to explore the challenges of deploying
MAI in public space. Initially, the technical challenges of deploying display
technology in public space have been summarised by Streitz et al. \cite{Streitz2003}. As the
design and implementation of digital media facades in the built environment progresses,
the purpose of such facades and the contextual characteristics of ''media
architecture" are addressed. Parameters that impact the integration of media
facades into the existing social fabric from a sociodemographic (environment),
technical (content) and architectural (carrier) perspective have been addressed by
Vande Moere and Wouters \cite{VandeMoere2012}. On the urban scale, the role of space, social
proximity and full body performative interactions in shared spaces have been
addressed by Fatah gen Schieck et al. \cite{Fatah2008}, O’Hara et al. \cite{Hara2008} and Peltonen et al. \cite{Peltonen2008}.

%\begin{figure}[htp]
%\centering
%\includegraphics[width=\textwidth]{IMG_3399Resized_Image_1024}
%\caption{ShareLaTeX logo}
%\label{fig:lion}
%\end{figure}
