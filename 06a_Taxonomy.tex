\chapter{Taxonomy for temporary media architecture}
\label{chapterlabel6a}

\section*{Chapter summary}

In this chapter I will .\newpage


\section{Taxonomy for temporary media architecture}


We would like to relate our findings to the initially examined interaction frameworks, which dealt with the (1) awareness space, (2) actor space, (3) action space and (4) physical space in the context of public displays or a media facade. 
In the following, we will discuss these frameworks in relation to the two design studies as outlined in the last section. 
We argue for the consideration of these multi-layered interaction spaces when designing interactive systems in urban spaces.
\textbf{Awareness space:} In both studies (VEIV London and SCSD Sao Paulo),
the direct interaction activities took place in the immediate vicinity of the TUI, whereas the focal and peripheral awareness spaces differed in their dimensions.
This is firstly due to the dissimilar size of the displays. The display used for the VEIV London study was only 2 by 6 m (i.e. 12 m2) compared to the large media facade for the SCSD Sao Paulo study with 3700 m2.  
At the same time, the difference in distance between the user interface and the display was significant: 5 m VEIV London compared to 33 m SCSD Sao Paulo. 
This seemed to have an impact on people’s spatial awareness with regard to dynamic displays in the city.
\textbf{Actor space:} Compared to the dynamic stream of pedestrians in the SCSD Sao Paulo study, the flow of people in the VEIV London study was rather calm, whilst the event was highly attended. 
The audience hardly changed during the party, but their actions altered; for instance, spectators turned into actors when walking up to the TUI to change the colour of the display and returned to their friends. 
Others simply enjoyed the presence of the colourful display or watched other guests interacting with it. 
In contrast, the dwelling time of the audience during the Sao Paulo event was much lower due to the high pass-through frequency of pedestrians on their way to the underground station.
\textbf{Action space:} The flow of people’s actions during the VEIV London event was different to SCSD Sao Paulo. 
Other than Michelis and Müller (2011) stated, the phases through which people had to go before they could actually interact with the MAI were not stringent in the VEIV London set up. 
People did not necessarily cross in the described order from (1) passing by to (2) viewing and reacting to (3) subtle interactions to (4) direct interactions to (5) multiple interactions and to (6) follow-up actions.
Instead, we observed groups gathering on the lawn walking up to the TUI to collectively explore the interactive system. In contrast to this, we identified human behaviour as explained by Michelis and Müller (2011) during the SCSD Sao Paulo study.
\textbf{Physical space:} The spatial layout in the VEIV London case consisted of a pedestrianised rectangular courtyard enclosed by the historical university buildings, compared to the vibrant two-directional avenue in Sao Paulo, which was lined by high-rise office towers.
In the VEIV London set up, there was hardly any \textit{gap space} in between the TUI and the display (i.e. ca. 5 m), compared to the congested main road in Sao Paulo, which divided the tangible interface from the
media facade (i.e. ca. 33 m). 
In addition, the comfort spaces, potential interaction spaces and social interaction spaces varied greatly in both studies. 
The enclosed courtyard and the positioning of the display within it released more of these spaces than the congested pavement in Sao Paulo.