\chapter{A framework for Media Architectural Interfaces (MAIs)}
\label{chapterlabel3}

\section*{Chapter summary}

In this chapter 

\newpage


\section{Designing for Interaction: Existing Interaction Frameworks}

Weiser (1991) and Dourish (2004) predicted a shift in personal computing and
interaction design. Interactions have moved from home and office desktops onto
mobile phones, tablets or public displays into urban spaces. Since then, extensive
research in HCI has explored social interactions mediated by public displays in
public spaces and brought several interaction frameworks forward.
Dalsgaard and Halskov (2010) have outlined the cases and challenges when
designing urban media facades and suggest considering eight challenges when
designing for media facades. The first two challenges focus on the need for new
interfaces that are required in urban settings as well as the integration of these into
the existing built environment. Based on this, a framework for designing complex
media facades has been developed by Halskov and Ebsen (2013), which includes
a description of what the difference is between media facades and conventional
displays. Scale, shape of display, pixel shape and their configurations as well as the quality of light were defined as parameters that differ when comparing media
facades with conventional displays. Most obvious is the fact that media facades are
not standardized in these properties, the scale is a way bigger than with conventional
displays, and compared to conventional displays media facades can have a
three-dimensional surface. In the following, we focus on four existing frameworks that we consider as important milestones towards an integrated media architectural design. Each
framework describes the relationship between humans and their action in the presence
of programmable electronic displays and in relation to the surrounding space.
These frameworks describe the following: (1) awareness space, (2) actor space,
(3) action space, and (4) physical space (Fig. 1).

\begin{figure}[htp]
\centering
%\includegraphics[width=\textwidth]{Background_frameworks}
\caption{Background frameworks}
\label{fig:lion}
\end{figure}

\subsection* {Awareness Space}
Research on awareness of public displays in relation to social interactions was first
described in HCI research, when noticing a novel social phenomenon around public
displays (Brignull and Rogers 2003). Three different types of ‘activity spaces’
were described: (1) Peripheral awareness activities: activities that take place in
the wider space around the display, where people socialize but are not necessarily
aware of the presence of the public display. (2) Focal awareness activities: in
this space, people are aware of the presence of the display. They are looking at the
display, discuss activities that take place around the display or learn how to engage
with the content. (3) Direct interaction activities: this is the space where individuals
or groups actively engage with the display. The research findings suggest that
people found it difficult to transit from one ‘activity space’ to another.
Later, Vogel and Balakrishnan (2004) published a spatial framework, which
described how users fluidly move from implicit interactions in the wider surrounding
towards explicit interactions when approaching the direct interaction space
around a public display.

\subsection* {Actor Space}
The actor space describes the different roles people take on when being in the
vicinity of interactive installations. By now, computers have moved away from
the desktop and novel interfaces appear, which spread into new spatial settings.
People are changing their role; in particular, in public settings people traverse
various awareness spaces, which afford specific roles. Consequently, a better understanding of what kind of roles these are, when and where people take them
on, is crucial for the design and deployment of interactive systems in urban spaces.
Reeves (2011) identified that the conventional user is actually changing her role
when passing-by, on looking or turning into a performer in the vicinity of an
interactive installation. Within this framework, the performing user plays the central
role when engaging with an interface. Her acting entices other people from
the audience into the experience; some of them may become performers. More
recently, Behrens et al. (2013), and Fatah gen Schieck et al. (2013), have explored
in detail how these roles are framed through the situated layout of urban screens.

\subsection* {Action Space}
In the last section, we clarify the diverse roles individuals take on in connection
with interactive installations; here, we describe the various phases of interactions
people traverse. The Audience Funnel by Michelis and Müller (2011) depicted a
framework that establishes a terminology for each transition. These phases were
identified as (1) passing by, (2) viewing and reacting, (3) subtle interaction, (4)
direct interaction, (5) multiple interactions and (6) follow-up action. Michelis and
Müller found that people proceed from one phase to the next in order to understand
the interaction. Boundaries are described as a series of thresholds that need
to be crossed before one can interact with a public display.

\subsection* {Physical Space}
The spatial localization of interactions is largely neglected in the work described
above. In contrast, Fischer and Hornecker (2012) have outlined an interaction
framework that focuses on the spatial properties of interactions. When looking into
the various encounter stages, ‘urbanHCI’ specifies different ‘interaction spaces’
on which people perform different activities. This includes the (1) display spaces,
(2) interaction spaces, (3) potential interaction spaces, (4) gap spaces, (5) social
interaction space, (6) comfort space and (7) activation space on which participants
behave differently towards an interactive media facade. Fischer and Hornecker’s
framework was explored through a media art project called ‘SMSlingshot’. This
project developed an interface that looks like a wooden slingshot, but its integrated
digital technology allows the user to shoot short text messages (SMS) together
with virtual paintballs on a media facade. Within these immediate spaces around
an electronic display (i.e. interaction spaces), passers-by stop, watch and start
playing with the media architectural interface and eventually change the look of
the media façade individually; others observe in groups from a distance, discuss
or engage as well; simultaneously, other pedestrians do not sense the presence of
such interactions and the existence of the media facade at all.
‘Screens in the Wild’, on the other hand, addressed the question of spatial layout
and its relation to technologically mediated interactions (Behrens et al. 2013;
Fatah gen Schieck et al. 2013). Four interactive and networked urban screens
have been deployed in four different neighbourhoods in London and Nottingham.
Within this approach, the following interaction zones were recognized: (1) direct
interaction space surrounding the display (direct); (2) the surrounding public
space (wide); and (3) across spatial boundaries, i.e. the remotely connected space
through the networked displays (networked). Sociospatial configurations mediated by urban screens were explored, and site-specific interactions were observed and
compared to more generic types of interactions. Indications were found that the
properties of the spatial layout play a significant role and, to a certain extent,
frame the type of interactions mediated through public displays (Fig. 2).

\begin{figure}[htp]
\centering
%\includegraphics[width=\textwidth]{DC_Behrens_FIGURE-02}
\caption{Taxonomy}
\label{fig:taxonomy}
\end{figure}

In summary, we plotted four existing spatial frameworks that describe interactions
between humans and displays from (1) awareness spaces, (2) actor spaces,
(3) action spaces and (4) physical spaces. Although three of the presented frameworks
dealt with smaller public displays, in comparison with the large urban
displays, juxtaposing these frameworks assists designers to understand the multilayered
design space when designing interactive systems for large programmable
displays. We contribute to this body of research by exploring specifically the
relation between TUI and large programmable displays (such as media facades)
in a given context. We clarify the notion of MAI and apply it on two design case
studies we conducted.

\section{ Media Architectural Interfaces (MAI) }

We introduce the notion of MAI. MAI capture an ecology of tangible (TUI) and
non-tangible interfaces. They can be considered as interactive systems in urban
space, which potentially entice people to step out of their routine and perceive
urban space or act differently within it. In more detail, we consider MAI as the
synthesis of situated and shared interfaces. They mediate participants’ engagement
with large programmable electronic displays such as urban screens, media facades
or media architecture. Tangible interfaces are generally located on street level,
whereas the connected displays are mostly vertical surfaces attached to buildings
or are the buildings themselves such as the case with media facades. Eventually,
they may disrupt movement and behavioural patterns in the given spatial setting.
The TUIs frame the interaction modalities (Müller et al. 2010) with the display
as well as they set the level of participation as described by Fritsch and Brynskov
(2011) and complemented by Caldwell and Foth (2014). Usually, these interfaces
call for explicit and shared interactions following the ‘urbanHCI’ framework
(Fischer and Hornecker 2012). Large programmable displays amplify participants’ interactions through the tangible interfaces, which depend on technical properties
of these displays such as type, shape, material, size and resolution of the display
(Halskov and Ebsen 2013).
Both interfaces (tangible and non-tangible) are dependent on the given sociospatial
setting (for example pedestrianized places, busy high streets or transport hubs).
As a consequence, the distance in between the tangible interface and the display
can vary. Further, when changing the properties within one of the three constituent
elements (i.e. the tangible situated interface, display or setting), the two other elements
are directly affected. This will be discussed in more detail in Sect. 5.
In the next section, we report on a case study, which describes an example of
a MAI consisting of two similar tangible user interfaces connected to two very
different displays within two different sociospatial settings. The aim is to test and
explore the notion of MAI and to eventually guide designers when developing
interactive systems for urban spaces.